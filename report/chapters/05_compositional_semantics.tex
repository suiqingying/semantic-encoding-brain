\section{组合语义与 supra-word 表征}

语言理解不仅涉及单词的意义,还需要基于上下文组合词汇产生超越字面含义的“超词”(supra-word)意义。Toneva 等(2022)\cite{Toneva2022}提出了一种数据驱动方法,利用 ELMo 模型\cite{Peters2018}的前向 LSTM 隐状态构建上下文嵌入,并通过线性回归消除单独词义的贡献,得到仅包含组合信息的残差嵌入。这种嵌入能捕捉隐含意义,例如“Mary finished the apple”中隐含的“吃完苹果”或“绿色香蕉”表示“未熟香蕉”,也可表示新语义组合。

将 supra-word 嵌入作为预测变量,研究者用其解释 fMRI 和 MEG 数据,发现经典词汇枢纽,如颞上回后部和颞下回,同样维护组合语义。此外,前颞叶也对 supra-word 嵌入敏感,说明词汇与组合语义在大脑中共享神经基础,需要前后颞区域协同维持。然而,他们在 MEG 数据中未检测到 supra-word 表征。这提示 supra-word 意义可能通过持续的、非同步的神经活动体现,而 MEG 对同步活动敏感,难以捕捉这种慢速信号。该结果强调不同脑成像技术对语义组合的敏感性不同,未来需要结合低频功率或相位同步等指标并使用多模态数据共同分析。

值得一提的是,supra-word 嵌入不仅揭示了组合语义的存在,还证明了分布式向量可以通过残差运算表示复杂的语义组合。这种方法与传统基于语法树的组合不同,它不依赖手工定义的规则,而是由模型在大量语料中学习到的统计规律。研究人员发现,将 supra-word 嵌入与词级嵌入结合,可以更准确地预测读者对隐含意义的理解程度。这一发现为改进语言模型提供了启示:未来可尝试不同的向量运算(如加权平均、向量差、张量积)以及递归或注意机制,来模拟人类如何积累和整合组合意义。此外,在神经数据中考察 supra-word 作用的时间动态是一个开放问题,未来可通过在 MEG 或 ECoG 数据中分析低频功率或相位同步来捕捉持续性组合信号,并结合工作记忆和注意任务,探讨语义组合与认知资源的关系。

除了 ELMo 生成的残差嵌入外,后续研究还探索了多种模型和运算来捕捉组合语义。例如,通过比较 GPT-2\cite{Radford2019} 或 BERT\cite{Devlin2019} 的上下文嵌入与各单词嵌入的向量差,可以获取另一种 supra-word 表征,这些表征在脑编码任务中的表现与 ELMo 残差相当。另一个方向是使用张量积或基于张量神经网络的方法,将两个词向量结合为高阶张量,从而显式表示交互项。虽然这些方法在计算上更昂贵,但它们提供了更丰富的组合信息。研究还发现,组合语义的表征不局限于双词短语,复杂句子中的嵌套关系和修饰语也可以通过递归地应用残差运算来分解。实验表明,这种递归嵌套的 supra-word 嵌入能够更好地预测人类对歧义句子的理解方式,说明模型对多义性消解有所把握。

组合语义不仅依赖词汇的线性组合,还与语法结构和语用知识紧密相关。在一些语言中,形态变化(如词序倒装、语气助词)会改变句子意味和语域,模型需要学习这些语法标记如何影响组合意义。此外,跨语言比较发现汉语、日语等语言的复合词和固定搭配大量依赖语法化的组合规则,这使得 supra-word 表征在这些语言中可能具有不同的统计特征。未来研究应扩展到不同语言的组合语义,检验模型能否捕获这些语言特有的组合规律。为了更精确地测量 supra-word 处理的神经时间动态,可以利用 ECoG 或高时间分辨率 fMRI,结合语音停顿和语调变化,分析大脑如何在听话者缓慢而连续的输入中形成语义组合。总之,supra-word 研究为我们揭示了词汇组合的复杂性,提醒语言理解是一个多层次、多维度的动态过程,需要模型和神经数据同时考虑语法、语义和语用因素。
