\chapter{跨语言与跨模态比较}

跨语言比较有助于检验模型与大脑匹配的普适性。Millet 等纳入英语、法语和普通话受试者,发现 wav2vec 2.0 自监督模型在不同语言中的层次映射非常一致:卷积层对应基本声学特征,中层对应语音特定特征,后期 Transformer 层对应语言特定信息。这一跨语言一致性表明,自监督模型捕捉了普遍的声学和语音规律,并且大脑对不同语言的加工共享相似的功能层级。

跨语言研究还应考虑语言类型学和文化差异。一些初步工作将语义地图方法应用于西班牙语、日语等非印欧语系,发现基本的语义区域位置类似,但某些文化特定概念(如礼貌等级、敬语、宗教词汇)在皮层中的激活强度存在差异。这可能反映不同语言在语义编码上的策略,以及语言经验对神经表征的塑造。此外,跨语言比较可以揭示不同书写系统对语义加工的影响。例如,表意文字(如汉字)阅读者更依赖视觉形状与语义联想,而表音文字阅读者更依赖声音规则。大脑在这两类文字的处理过程中激活的视觉和语音区域有所不同,这些差异应在模型评估中加以考虑。未来扩展到双语者和方言使用者,将有助于理解语言经验对语义系统的可塑性和适应性。

跨模态比较则揭示不同成像技术对语言处理的敏感性差异。Schrimpf 等的集成建模同时分析了 fMRI 和 ECoG 数据,发现 Transformer 模型对两种模态的预测能力高度一致。然而,Toneva 等发现 MEG 数据无法检测 supra-word 表征。这些对比强调,需要结合多模态数据以捕捉大脑语言处理的不同时间和空间尺度。例如,fMRI 能捕捉慢速血氧反应,适合发现持续性语义信息,而 MEG 则对快速电同步更敏感,适合捕捉即时处理。这些差异需在模型与大脑对齐时仔细考虑。

跨语言研究还强调了语言类型学的多样性对语义系统的塑造作用。黏着语(如土耳其语、芬兰语)通过在一个词上附加多个语素表达语法关系,大脑可能利用这些形态学线索在早期阶段预测词干和附加成分的组合。屈折语(如拉丁语、俄语)则通过词形变化表示时态和格标记,需要跨更长距离的依存关系解析。孤立语(如越南语、泰语)依靠语序和功能词表达语法,使得句子结构的预测依赖于对短语层级的掌握。这些差异影响了语言模型和大脑在预测下一词时使用的策略,也要求我们在模型评估中纳入多样的语言。

文化因素也通过语言使用频率、隐喻习惯和礼貌策略等影响语义系统。例如,某些文化中对亲属关系有精细的词汇区分,而在另一些文化中对颜色或味觉的词汇更为丰富。研究者在跨文化采样时发现,与家庭和情感相关的词在东亚文化中引发的皮层激活更为广泛,而与个人独立性相关的词在西方文化的前额叶激活更强。这些差异提醒我们,语义地图具有可塑性,会随着语言环境和社会规范调整。未来应在不同文化群体中重复语义映射和脑编码实验,构建跨文化的语义参考图谱。

跨模态比较的意义不仅在于技术差异,还在于不同信号反映的神经过程各有侧重。ECoG 捕捉皮层表面电位,能实时反映快速同步活动,适合研究词汇和音素边界的瞬时处理;fMRI 捕捉血氧变化,反映几秒内的总体活动,适合研究持续的语义和情节加工;MEG 介于二者之间,反映大规模神经群体的同步,但对深层结构的敏感性较低。通过在同一受试者身上同时采集或跨实验结合这些数据,可以构建时间分辨率和空间分辨率兼顾的动态语义模型。例如,可先利用 MEG 确定 supra-word 组合发生的时间窗,再利用 fMRI 确定具体位置。跨模态融合将带来更全面的语言加工图景,促进理论与模型的发展。

跨语言语义映射的进一步研究需要涵盖非印欧语言群体以及低资源语言,如非洲的班图语系、美洲的纳瓦荷语或澳大利亚的皮京语。这些语言在语音、形态和句法结构上具有独特特征,如点击音、序列化动词或双数标记,可能导致不同的语义组织模式。通过将这些语言纳入语义地图项目,可以检验语义系统的普遍性,并为语言多样性保护提供科学依据。此外,随着语言接触和全球化的深入,多语言者的大脑可能展现出更加灵活的语义网络和预测策略。研究显示,精通多种语言的人在切换语言时能够快速调整语义激活模式,这反映出一种在语义空间中的动态抑制与增强机制。将这种多语灵活性纳入神经模型,不仅有助于理解双语脑如何管理多个词汇和语法系统,也能启发开发能够动态切换语境的人工模型。未来的跨语言研究应将样本扩展到不同的社会群体,包含方言和混合语言,如新加坡英语和斯普兰语,从而捕获语言演化和创新对语义系统的影响。
