\chapter{逆向工程:从神经信号重构连续语义}

除了“编码”(从刺激预测脑反应),近年来研究也开始系统推进“解码”(从脑反应重构刺激)与“逆向工程”。Tang 等(2023)\cite{Tang2023}展示了利用非侵入式 fMRI 在连续叙事场景下重构语义内容的可行性:模型并非逐字逐句复原原文,而是重构与原叙事在语义上高度相近的表达。这一结果提示,大脑在自然语境下的表征更接近“意义层级”的压缩,而不是对词形的逐点记录。

在方法上,解码任务常被形式化为寻找最可能的文本序列 $S$:
\[
\hat{S}=\arg\max_{S} P(S\mid R)\propto \arg\max_{S} P(R\mid S)\,P(S),
\]
其中 $R$ 表示脑反应(如 fMRI 体素时间序列),$P(R\mid S)$ 对应“编码模型”,而 $P(S)$ 则提供语言先验(可由语言模型给出)。在实际求解中,研究者通常在候选空间内采用波束搜索(beam search)等策略,平衡“脑一致性”与“语言可行性”。这类工作也带来重要的科学与伦理问题:它既为检验语义系统的分布式表征提供了新的量化指标,也推动了对脑数据隐私与可解释性的讨论。
