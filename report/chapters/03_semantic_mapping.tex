\chapter{语义系统的脑映射与分布式表征}

\section{数据驱动绘制语义地图}
在自然语言理解中,大脑对语义信息的表示是分布式的。Huth 等(2016)\cite{Huth2016}通过让受试者聆听长达两个小时的自然故事,利用词共现构建的 985 维词嵌入和体素级正则化回归,绘制了语义系统的高分辨率地图。模型经交叉验证可预测新故事的 fMRI 响应,说明嵌入捕获了稳定的语义特征。通过对模型权重进行主成分分析,他们提取出四个跨受试者共享的主要语义维度,并使用 PrAGMATiC 算法将这些维度投影到皮层,发现左半球约含 77 个语义区域,右半球约 63 个。这些区域不仅涉及传统语言区,还扩展到顶叶皮层(LPC)、内侧顶叶(MPC)和前额叶的默认模式网络(DMN)区域,其中中心区域偏向社会和人物概念,外周区域偏向数字、视觉或触觉概念。这一发现打破了仅有左侧优势的传统观点,揭示语义系统在左右半球间较为对称。

除绘制地图本身外,Huth 等还详细描述了模型的训练和评估流程。他们构建了 985 维语义特征矩阵,其中每一维度表示英语词与语料中其他词的共现概率。为了消除低级听觉因素,模型在回归时加入词率、音素率和声学特征作为协变量。在训练阶段,研究者利用 10 折交叉验证评估模型对新故事的预测能力,保证了结果不依赖特定刺激。这种严格的验证使绘制出的语义地图具有可重复性和泛化性。通过观察各语义维度在皮层表面的分布,他们发现概念的抽象程度呈后—前梯度:背侧和后侧区域偏向具体感官相关概念,如动作、视觉和听觉;腹侧和前侧区域则偏向抽象、社交和情感概念。该渐变跨越颞叶、顶叶和前额叶,反映大脑可能按抽象度或表象类型组织语义信息。作者建议,这一渐变与从知觉到抽象思维的连续加工路线相一致,为理解概念结构提供了新的神经学证据。

这种数据驱动的语义映射方法具有多重意义。首先,它利用自然故事这一生态刺激,克服了传统实验采用单词或短语的限制,展示出在真实语境下绘制语义地图的可行性。其次,语义地图可作为参照框架,便于不同研究之间比较语义表征位置。作者指出,未来研究需改进区域划分算法,兼顾离散区域与功能渐变。此外,通过跨语言和跨文化采样,可以检验语义系统的普遍性和可变性。

基于自然故事的语义映射提供了更全面的视角。一些后续研究将 Huth 等的方法扩展到其他语言,如西班牙语、法语和中文,发现语义地图的宏观结构具有高度一致性,这表明大脑语义系统的组织具有跨语言普遍性。然而,在某些文化特定概念上,如颜色词、亲属称谓或食物名称,不同语言的激活模式存在细微差异,这可能与语言中相关类别的词汇丰富度或社会文化重要性有关。通过比较不同语言的语义地图,研究者可以揭示概念表示的文化可塑性及其神经基础。

语义地图还揭示了默认模式网络在语义处理中的核心地位。DMN 包含的角回、后扣带皮层和内侧前额叶长期以来被认为参与自发思维、内省和记忆检索。Huth 等的结果显示,这些区域也积极参与语义加工,尤其是涉及社会、情感和思维推理的概念。这一发现使我们重新审视 DMN 的功能定位,提示它可能不是专门处理“与任务无关”的思维,而是在语义推理过程中发挥中枢作用。语义地图的结果还打破了传统认为左半球主导语言的观点,揭示两半球在语义任务上的对称性。不过,由于 fMRI 在前颞叶的信号噪声较大,某些语义区域可能仍未被发现,因此需结合 ECoG 或高场强 fMRI 提高空间分辨率。

在对语义地图的后续分析中,研究者进一步解析四个主要语义维度分别对应哪些概念群。例如,第一个维度从有生命的生物到无生命物体,反映动—静连续体;第二个维度从社会交往到工具使用;第三个维度从视觉场景到感知属性;第四个维度从数量与空间到情感与心理状态。通过比较这些维度在皮层上的渐变,发现背侧区域偏向具体感官经验,腹侧区域偏向抽象社会知识。实验还比较了不同故事段落中语义向量的变化,结果显示语义维度的激活模式随故事推进而动态演化,表明语义处理具有时间依赖性而非静态。作者提出,可以将语义地图作为生成刺激的指南,选择刺激中激活特定区域的词语或句子,研究这些区域对语义推理的因果作用。

语义地图不仅揭示出概念在皮层表面的分布,还呈现出随着故事语境变化而动态更新的模式。后续研究利用滑动窗口技术,分析语义维度的激活随时间的变化,发现故事高潮时期社会和情感维度的激活显著增加,而叙述背景期则更多激活物体和场景维度。这表明大脑语义表示具有时间敏感性,会根据故事进程调整重点。另一些研究通过比较不同叙事体裁,如对话、新闻报道和诗歌,发现诗歌中的抽象情感维度激活更强,而对话和新闻则更依赖社会互动维度。除此之外,跨语境分析显示,同一概念在不同故事中的激活模式可能不同,反映语义表示与篇章背景的耦合。未来可以结合自然语言生成模型,控制故事内容和风格,系统探索语义表示的动态调节。进一步地,利用联结梯度分析,可以描绘语义网络在皮层内部的连续变化,揭示从感觉相关区域到抽象思维区域的渐变。这种多维度、多时间尺度的语义地图将为理解语言语义的动态生成提供新的视角。

\section{语义类别与关系的分布式编码}
在对语义地图的进一步探索中,Zhang 等(2020)\cite{Zhang2020}让受试者听 11 小时故事,建立体素级编码模型,预测数千个单词的脑响应。他们发现,大脑并不以离散模块表示不同语义类别,而是通过广泛重叠的区域同时编码多种类别。例如,工具类概念在左侧顶下小叶、后中颞回和颞上回均有表示;交流与情感等抽象概念则在右前颞区和顶叶更为明显。通过分析词汇的具体性,研究者观察到左半球偏向具体、感官相关概念,而右半球更偏向抽象、内省相关概念。这一左右半球差异揭示语义系统在处理不同类型概念时的功能特化。

不仅语义类别,概念之间的语义关系也可以映射到皮层。Zhang 等利用词向量差表示语义关系,例如整体—部分、类—属、对象—属性等,并构建关系编码模型。他们发现,“整体—部分”关系在默认模式网络区域(如角回、后扣带皮层)呈现明显激活,而前顶叶注意网络呈现抑制;其他关系则在不同网络中表现不同的激活抑制模式。这种共同激活与抑制的模式表明,大脑通过协同的功能网络而非独立区域编码语义推理。作者还发现语义关系的网络模式与语义类别脱钩,例如“手—手指”和“动物—动物园”属于不同类别但具有相似的关系模式。这表明大脑可能存在专门处理抽象关系的网络,与 DMN 中的思维漫游或内省功能相联系。

值得进一步说明的是,Zhang 等划分的九个语义类别包含工具、人类、植物、动物、地点、交流、情感、变化和数量等,每个类别又由数百个单词组成。对这些类别的分析显示,具体概念(如“锤子”“狗”“苹果”)在多感官和运动相关区域呈现较强激活,而抽象概念(如“自由”“希望”“交流”)则在前额叶和顶叶默认模式网络表现更强。语义关系方面,除了整体—部分、类—属和对象—属性,研究者还考虑了名词与动作之间的关系、时间关联、空间关联和事件因果关系。他们利用 SemEval-2012 评测集中的句子对构建差向量,并利用这些向量预测脑响应。结果表明,不同关系在皮层上呈现高度一致的空间模式,说明大脑可能通过共享的网络处理各类关系推理,而不是为每种关系单独配置区域。这一发现拓展了我们对语义系统功能的理解,表明抽象关系加工与默认模式网络的内在思维密切相关,并可能涉及对情境和情感的整合。

 Zhang 等进一步分析了不同类别词汇在皮层中的精细分布。例如,人类和动物概念不仅在后颞与顶叶区域活跃,还在视觉皮层的被动激活区出现,可能与想象或回忆相关;情感和交流类词汇在角回、内侧前额叶等 DMN 区域更强,说明这些抽象概念与自我反省和社会认知密切相关。研究者还比较了不同关系类型,如“物品与使用者”“因果与结果”“部分与整体”,发现关系向量激活模式的相似性与关系的逻辑结构相关。例如,因果关系与时间关系的皮层模式相近,体现故事中事件连贯性在脑中的共通表示;而反义关系激活的网络更分散,可能需要更多注意和工作记忆资源。这些发现提示语义关系不仅是词义差向量,在大脑中也体现为跨网络的协同模式。

\section{地图与网络的意义}
语义地图和语义关系研究表明,语义系统既有局部分区又存在跨区域的连续功能梯度。Huth 等的 PrAGMATiC 算法划分出多个语义区域,但假设每个区域内部同质,这难以捕捉某些功能渐变。未来需要发展既能识别离散区域又能描述功能渐变的模型,如连通性梯度分析。Zhang 等的研究强调语义关系的网络化特点,通过默认模式网络与注意网络的协同活动编码抽象关系。这些发现提示,语义认知不仅依赖单个区域的选择性,也依赖跨网络的动态交互,理解语言中的推理和抽象思维需从网络视角入手。

通过结合皮层连接信息,研究者建议语义系统可划分为若干功能网络:中心的 DMN 支持抽象概念和情景推理,背侧注意网络支持概念的检索和选择,腹侧语义网络支持具体物体和动作信息。语义关系的编码往往跨越这些网络,例如“部分—整体”关系需要同时激活 DMN(处理整体概念)和抑制顶叶注意网络(抑制无关信息)。这表明语义处理可能涉及网络间的抑制与兴奋平衡。未来需要采用动态图模型或有效连接分析,理解在语义推理过程中网络交互的因果顺序。此外,语义网络与其他认知网络如工作记忆、情绪和奖励系统的交互也值得探索,因为实际语境中的语言理解往往伴随情感体验和行动决策。
