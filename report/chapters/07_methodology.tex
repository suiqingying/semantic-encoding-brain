\section{方法论:编码模型、评价指标与统计检验}

\subsection{编码模型:岭回归与分带岭回归}
当以高维模型表征作为特征时,脑编码模型往往面临 $P\gg N$ 的病态问题,因此线性回归通常配合正则化使用。最常见的形式是岭回归:
\[
\hat{\beta}=(X^\top X+\lambda I)^{-1}X^\top y,
\]
其中 $X$ 为特征矩阵、$y$ 为神经响应、$\lambda$ 为正则化系数。在特征由多子空间构成(例如声学特征与语义特征)时,还可使用分带岭回归(banded ridge regression),为不同子空间分配不同的正则化强度,从而进行方差分解并更清晰地区分不同信息源的独立贡献。

\subsection{噪声天花板与显著性检验}
由于神经测量本身存在噪声,模型预测的相关性上限受到“噪声天花板”(noise ceiling)约束。实践中常以重复试次的一致性或跨被试一致性估计可解释方差,从而避免对模型性能的过度解读。对单体素或单通道的显著性判断,则常结合置换检验(permutation test)构建零分布,以控制多重比较并提升结论的稳健性。
