\section{讨论:综合视角与未来方向}

\subsection{预测还是特征学习?}
关于神经语言模型在大脑编码任务中成功的解释存在两种观点。一方面,集成建模显示模型的脑拟合度与下一词预测准确率高度相关。这一结果常被解读为大脑优化于预测未来输入。另一方面,层级分析与通用迁移指标则指出,模型的预测能力不一定决定脑编码性能,中间层或迁移能力更能解释大脑数据。因此,我们需要超越简单的二分法,认识到预测和特征学习可能共同作用:大脑在处理语言时既利用预测生成候选,还依赖丰富的统计结构进行解析。未来研究可以通过操纵模型的训练目标(例如仅训练下一词预测 vs. 联合其他任务)来评估这些因素在脑编码中的相对贡献。

\subsection{分布式语义系统与网络化视角}
语义地图和语义关系研究揭示,大脑语义系统并非由少数专门区域组成,而是由广泛的分布式网络构成。PrAGMATiC 地图揭示 DMN 内有多个语义亚区,且左右半球分布对称。Zhang 等进一步发现语义关系通过 DMN 的激活与前顶叶注意网络的抑制共同编码,说明抽象推理依赖网络的动态交互。理解语义系统需要关注这些网络的连接模式和时空动态,而不仅是个别皮层区域的选择性。未来可以利用功能连接分析或图论方法,探索语义网络在不同任务和状态下的重构与调节。

\subsection{模型特性与大脑对应}
神经模型的学习机制和架构对与大脑的对应性具有重要影响。自监督语音模型在少量未标记数据上就能学得与大脑相似的层级表示,强调自监督和对比学习的价值。Transformer 的注意头分析提供了更细粒度的功能对齐视角,揭示不同层和头在皮层中的对应关系。这些方法说明,大脑和模型可能共享某些计算原则,如局部与全局整合。但目前模型仍缺乏对生理约束的考虑。加入时间连续性、能量限制、发展过程等约束,可能使模型更符合大脑。

\subsection{组合语义的新挑战}
supra-word 研究指出,组合语义在大脑中与词汇语义共享神经基础但依赖不同的维持机制。当前的语言模型通常通过固定窗口产生静态嵌入,难以捕捉组合语义的动态特性。未来应探索结构化或递归模型,在模型中显式实现组合运算,并比较不同组合算法对脑数据的预测能力。此外,需要在 MEG 和 ECoG 中寻找捕捉组合语义的合适信号,例如低频功率或相位同步。

\subsection{跨文化与多语言的扩展}
现有研究主要使用英语和西方语言,受试者背景也相对单一。Huth 等指出不同个体间语义地图的相似性可能源自共同的生活经验;Zhang 等则发现抽象概念与右半球关联的强度可能受文化差异影响。因此,未来需要在不同文化和语言环境中采集数据,比较语义系统的共性与差异。研究应扩展到儿童、双语者或方言使用者,以揭示语义系统的发育和可塑性。这将帮助我们理解语言经验与神经组织如何交互,促进构建具备跨语言普遍性的模型。

讨论未来方向时,还需考虑神经语言模型与其他认知系统的交互。语言理解往往伴随记忆检索、情感评价和动作规划等过程,语义系统必须与海马回、杏仁核以及额顶网络协同工作。现有模型多数仅处理语言输入,缺乏与视觉和情感系统的互动。将视觉和情感通道融入模型,可以模拟故事理解中的场景想象和情绪反应。例如,电影描述不仅包含语言,还伴随画面和音乐,这些多模态刺激触发更复杂的语义及情感网络。开发能够同时处理文本、图像和声音的多模态模型,并用其预测多模态脑数据,将为全面理解自然语境下的语言加工奠定基础。

另一个重要方向是发展因果推断的方法。目前多数研究基于相关分析,难以确定模型表示对大脑活动的因果作用。结合经颅磁刺激、脑损伤研究或神经反馈,可以测试某些模型表征是否必要或充分。例如,当模型预测出高语义突发点时,相关脑区是否必然会出现因果响应?机器学习中的可解释性工具(如特征重要性和层次可视化)也可与神经干预结合,验证特定特征在语义加工中的作用。通过这些方法,我们可以揭示模型与大脑之间的因果联系,而不仅是相关性。

此外,伦理和公平问题应贯穿研究全过程。语义系统与价值观和社会经验密切相关,训练数据中的偏见可能导致模型学习到有害的语义关联,例如性别刻板印象或种族偏见。在比较不同文化和语言时,也必须尊重多样性并避免文化中心主义。开发公平、透明且可解释的语言模型,并理解其与大脑语义系统的异同,将有助于构建包容、安全的人工智能系统。


另一个值得关注的领域是儿童语言学习与成人语言理解的差异。婴幼儿在缺乏明确监督的情况下,通过与环境互动自然习得语言,其学习过程可能更依赖于语音和语义的统计共生,而非明确的预测。研究表明,儿童的大脑在处理语言时更依赖音节层级和韵律模式,而成人则能利用句法和语义进行更高级的预测。当前神经语言模型主要基于成人语言语料训练,忽视了语言发展阶段的差异。未来可以通过训练自监督模型在儿童对话语料上,模拟语言的发展过程,并与不同年龄段的脑成像数据比较,揭示语言系统的成熟轨迹。此外,还应关注老年人的语言加工变化,研究老化和神经退行性疾病如何影响语义网络和预测机制。通过跨年龄的比较,我们可以建立更加全面的语言认知模型,为教育和康复提供科学依据。

从更广阔的角度看,语言研究与社会科学、哲学和人工智能伦理密切相关。语言不仅是沟通工具,也是思想和文化的载体。神经语言模型的广泛应用有可能影响公共舆论和社会认知结构,因此研究者需要审慎评估模型的社会影响。例如,语言模型生成的内容可能强化现有偏见或创造新的误解;模型的预测机制可能在某些文化中被误解为超越或取代人类思考。通过与哲学家、社会学家和伦理学家的合作,可以更深入理解语义系统与社会结构的相互作用,并制定适合不同文化环境的技术规范。科学家还应通过科普教育,让公众理解神经语言模型的能力与局限,建立合理期待,避免神话化技术。综观全局,语言的神经与计算研究既是科学探索,也是社会实践的一部分,必须兼顾科学价值和社会责任。
