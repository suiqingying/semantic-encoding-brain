\section{引言}

近年来,随着深度学习技术的突破,人工神经网络在自然语言处理(NLP)领域获得了前所未有的成功。与此同时,认知神经科学的发展使得我们可以通过功能磁共振成像(fMRI)、电生理记录(ECoG)或脑磁图(MEG)等技术测量大脑在理解语言时的动态响应。将这两条研究路径结合起来,比较神经网络模型的内部表征与大脑活动,既有助于解释模型为何有效,也有助于揭示人脑处理语言的规律。特别是语言模型的训练目标往往是预测下一个词,这与某些理论认为大脑通过预测未来输入来减少处理负荷的观点不谋而合。然而,模型解释大脑的成功是否意味着大脑真的在实施同样的预测机制,仍存争议。

另一方面,数据驱动的语义地图绘制为理解语义系统提供了全新的视角。在自然语言刺激下,大脑的语义区域如何分布?语义类别和语义关系如何在皮层中表示?这些问题的解答不仅深化我们对语言认知的理解,也可能反过来指导人工模型的改进。本综述不按单篇文章分别介绍,而是围绕核心主题整合现有研究成果:首先概述预测编码理论及其在语言领域的支持与质疑;随后总结利用神经语言模型建立大脑编码模型的集成建模工作;接着讨论语义系统的脑映射,包括语义类别与语义关系的分布式表征;之后阐述不同学习机制下模型与大脑的对齐,如自监督语音模型和 Transformer 的注意头分析;进一步介绍组合语义(supra-word)表征的发现;最后探讨跨语言、跨模态比较的意义,并在讨论部分提出未来研究方向。通过这种综合视角,我们旨在呈现大脑语言处理与人工神经模型互动的完整景观,并指出当前研究的关键问题和潜在路径。

本领域的交叉研究不仅推动认知科学的发展,也对人工智能的设计产生深远影响。一方面,神经网络模型已从仅追求任务性能的“黑箱”演进为可以解释人类数据的认知模型。通过将脑成像数据纳入模型评测,研究者得以揭示模型所学的语言表示是否具有生物学真实性,从而指导模型架构和训练目标的调整。另一方面,大脑科学家利用模型生成具备特定语言属性的刺激,设计实验以验证大脑的语义组织、加工顺序和层次结构。例如,模型可以产生具有不同预测难度或不同语法依存结构的句子,用于探索大脑在预期违背或结构复杂度下的反应模式。正如后文所述,这种双向互动正在形成新的研究范式,促使人工智能和神经科学共同迈向更高水平的理解。

近年来,随着脑成像技术的进步,研究者不再局限于通过行为数据推断认知过程,而是能够在毫秒乃至更高时间分辨率上观察大脑对语言刺激的响应模式。结合深度学习模型,这种方法提供了一个跨尺度的桥梁,使我们能同时访问神经元群体的活动和算法级别的信息。与此同时,人工神经网络的内部表征从简单的线性结构发展为复杂的层次化系统,这些系统在训练过程中自发产生语法树状结构、意象表征乃至对世界的统计认识。对比这些表征与脑成像数据,可以揭示哪些结构是普遍存在于人类语言认知中的通用属性,哪些是模型特定的产物。

此外,围绕人工智能的伦理与实际应用问题也提醒我们,理解模型如何与人类认知对齐具有重要意义。语言模型被广泛用于教育、医疗和政策决策等领域,若忽略其与人类理解方式的差异,可能导致误解或偏差,甚至出现严重的社会后果。通过研究神经模型与大脑的对应,我们可以识别模型在语义推理、情感理解或常识推理方面的缺陷,并制定改进策略,例如在训练数据中引入多样化的语境、增加现实世界知识或限制模型的信任范围。总之,人工智能与神经科学的交叉研究不仅拓展科学边界,也对社会和伦理产生深远影响。

值得指出的是,语言不仅是词序列的累积,还包括丰富的形态学和语用学信息。许多语言使用屈折词缀、变化规则和音调来表达语法关系和时态、体、态等语义特征。例如,阿拉伯语的三根辅音根通过元音变换表达不同的词义,因纽特语的多重粘着导致单个词即可表示复杂句子。这些形态学特性为预测提供了额外线索:在富形态语言中,词干和词缀的组合给出句法框架,大脑可能利用形态学规则来缩小下一词的候选范围。未来的神经模型应考虑形态学多样性,不能仅依赖英语语料,否则在解析屈折或黏着语言时可能无法对齐大脑处理。随着研究扩展到更多语言,比较不同形态类型的大脑活动将揭示语言特性对语义系统塑造的影响。

此外,语言的句法结构也对预测机制和语义分布有显著影响。语序固定的语言(如英语)与语序自由的语言(如拉丁语或俄语)在理解过程中的策略不同:在语序固定的语言中,听者和读者可以依赖固定的位置来识别主语、宾语和谓语,而在语序自由的语言中,需要依靠格标记和语义角色进行解析,这可能增加理解负荷并影响预测的层级。此外,汉语等话题优先语言常通过主题—述题结构组织信息,大脑必须在句子前段确定主题并预测后续信息。在某些美洲原住民语言中,主谓宾顺序可以灵活改变以突出新信息或对比,这种突出与焦点结构需要更复杂的语用推理。语言模型和脑编码研究应进一步探索语序变异如何影响预测和语义系统的组织,通过控制句法自由度和引入格标记等操作,比较不同语言的神经反应模式。
