\section{数据、对齐表与 TR 级刺激构建}
本项目使用的原始文件位于 \texttt{data/raw/}。fMRI 数据以 ROI 形式预先整理为 \texttt{21styear\_all\_subs\_rois.npy},对齐表位于 \texttt{21styear\_align.csv},音频刺激位于 \texttt{21styear\_audio.wav}。\texttt{src/data.py} 将这些文件加载为可被特征抽取与编码建模直接使用的结构:\texttt{load\_fmri()} 返回一个以被试编号为键的字典,每个条目是形状为 $(T,360)$ 的矩阵,表示 $T$ 个 TR 上 360 个 ROI 的 fMRI 信号;\texttt{load\_audio()} 以固定采样率读取整段音频;\texttt{load\_align\_df()} 读取对齐表并为每个词构造 TR 编号。

对齐表 \texttt{21styear\_align.csv} 每行包含四列:保留大小写的词、全部小写的词、词开始时间戳(秒)与词结束时间戳(秒)。对齐表中存在缺失项,代码对时间戳进行向后填充,并将缺失词以 \texttt{none} 作为占位。随后根据 TR 时长将词级时间戳映射到离散 TR 索引。项目设置 TR 为 1.5 秒,因此对于词开始时间 $t$(单位秒),对应 TR 索引为 $\lceil t/\mathrm{TR}\rceil$。这一步的输出是一个包含 \texttt{tr} 列的数据框,它把每个词归入某一个 TR,从而为后续“把词级特征聚合为 TR 级特征”提供了确定的分组键。

TR 级刺激构建需要同时处理三条时间轴:词级时间轴(用于文本与文本端对齐)、连续波形时间轴(用于音频分片)、TR 采样时间轴(用于与 fMRI 对齐),以及 BOLD 延迟轴(用于 FIR 延迟展开)。本项目对文本与音频采取统一的策略:先在原始粒度上抽取预训练模型特征,再将特征聚合到 TR。对于文本而言,\texttt{src/text\_pipeline.py} 先为每个词构造上下文窗口(默认 200 token),输入语言模型得到词级或 token 级表征,随后按 \texttt{tr} 分组,对同一 TR 中所有词的表征求均值得到 TR 级文本特征。运行时可能出现 pandas 的 FutureWarning,这属于 API 行为变更提示,不影响当前版本下的数值计算与输出文件。

对于音频而言,连续波形根据 TR 窗口切分为一系列 chunk。配置 \texttt{src/config.py} 中的 \texttt{AUDIO\_SR=16000} 表示采样率为 16kHz;当窗口设置为 1TR、2TR、3TR、6TR 时,分别对应 1.5s、3.0s、4.5s、9.0s 的音频片段。每个片段作为一个输入样本送入音频模型得到表示,形成与 TR 一一对应的序列。窗口长度不仅决定了音频表征是否覆盖跨 TR 的韵律与语音单位结构,也会与 FIR 延迟展开共同决定“刺激历史覆盖范围”,因此音频模型部分会系统比较不同 TR 窗口的效果。

为了使后续编码模型稳定训练,特征在进入回归之前会进行降维。当前实现默认使用 PCA 将 TR 级特征降到 250 维(\texttt{DEFAULT\_PCA\_DIM=250}),其目的在于减轻高维特征与有限样本量组合导致的病态问题,并降低回归求解成本。所有预处理都在特征与脑信号完成 TR 级对齐之后进行,从而保证特征矩阵与 fMRI 的时间轴严格一致。
