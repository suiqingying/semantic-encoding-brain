\chapter{多模态模型结果:Whisper}
多模态模型部分的结果来自 \texttt{results/summary.csv} 中 \texttt{/multimodal/} 相关条目以及对应目录下的 \texttt{corr\_layer*.npy}。图 \ref{fig:multi_best} 展示了已完成的多模态模型在最佳层上的全脑均值相关,图 \ref{fig:multi_tr_trend} 展示了不同 TR 窗口下的最佳层趋势。当前结果中,\texttt{openai/whisper-base} 在 6TR、layer2 达到 0.0889,已经非常接近强音频基线 Wav2Vec2 的 0.0895;\texttt{openai/whisper-small} 在 6TR、layer9 达到 0.0844。值得注意的是,Whisper-base 在 1TR、2TR、3TR 条件下也有多层完成记录,其最佳均值分别为 0.0270、0.0408、0.0579,趋势与音频模型一致,说明多模态模型同样高度依赖较长时间窗的聚合。

从“是否优于音频基线”的角度来看,当前已完成的多模态结果并未显著超越音频最优(WavLM 0.0916),但 Whisper-base 已经达到与 Wav2Vec2 接近的水平。由于本报告不引入未完成的显著性检验或噪声天花板估计,因此在解释上保持克制:可以确认多模态模型在当前设置下具有较强的可预测性,但不能仅凭均值相关就断言其“比音频更语义化”或“更接近高阶语言区”,这些需要结合 ROI 命名映射与进一步统计。

与文本与音频章节一致,本章节同样提供“类别脑图对照”与“最优模型脑图”两类图像。图 \ref{fig:multi_montage} 将 Whisper-base 与 Whisper-small 在各自最佳配置下的 corr map 并置,以便观察不同模型在空间分布上的共性与差异。图 \ref{fig:multi_whisper_map} 展示 Whisper-base 最优配置(6TR,layer2)的单独脑图,图 \ref{fig:multi_whisper_roi} 展示其 ROI Top20,用于与音频最优模型的 ROI Top20 做直接比较。由于 \texttt{results/roi.csv} 当前只记录 ROI 编号而未提供解剖名称映射,本报告在区域解释上不引入超出文件所能支持的命名推断,而是把重点放在可复现的数值与分布差异描述上。

\InsertFig{multimodal_best.png}{0.86\linewidth}{多模态模型最佳层的全脑均值相关系数(由 report/scripts/make\_figures.py 从 results/summary.csv 生成)。}{fig:multi_best}
\InsertFig{multimodal_tr_window_trend.png}{0.92\linewidth}{多模态模型在不同 TR 窗口下的最佳层性能趋势(由 report/scripts/make\_figures.py 从 results/summary.csv 聚合生成)。}{fig:multi_tr_trend}
\InsertFig{multimodal_class_montage.png}{0.96\linewidth}{多模态模型类别脑图对照:Whisper-base(6TR,layer2)与 Whisper-small(6TR,layer9)(由 src/run\_plot\_corr\_maps.py 绘制并组合)。}{fig:multi_montage}
\InsertFig{multimodal_openai_whisper-base_6TR_corr_layer2.png}{0.92\linewidth}{Whisper-base(6TR,layer2)相关图可视化(由 src/run\_plot\_corr\_maps.py 从 corr\_layer2.npy 绘制)。}{fig:multi_whisper_map}
\InsertFig{roi_whisper6tr_l2.png}{0.86\linewidth}{Whisper-base(6TR,layer2)对应 corr map 的 ROI Top20(由 report/scripts/make\_figures.py 从 results/roi.csv 生成)。}{fig:multi_whisper_roi}
