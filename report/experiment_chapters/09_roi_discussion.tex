\section{ROI 分析与综合讨论}
ROI 分析的目标是回答“不同特征在皮层不同脑区的预测性能是否存在系统差异”。在当前工程实现中,fMRI 已经以 HCP-MMP 360 ROI 的粒度保存并用于回归,因此 ROI 分析不是从顶点或体素再聚合到 ROI 的二次统计,而是对已保存的 ROI 相关向量进行整理与排序。具体流程为:\texttt{src/run\_roi\_analysis.py} 扫描 \texttt{results/} 下所有 \texttt{corr\_layer*.npy} 文件,将每个 corr map 的 360 个相关值与其源路径一起写入 \texttt{results/roi.csv};随后 \texttt{report/scripts/make\_figures.py} 读取 \texttt{results/roi.csv} 与 \texttt{results/summary.csv},为若干代表性配置绘制 ROI Top20 条形图。由于 \texttt{results/roi.csv} 明确记录了每个 ROI 值对应的源文件路径,因此任何一张 ROI Top20 图都可以追溯回唯一的 corr map 文件,实现与统计图、脑图一致的可复核链路。

从当前三张 ROI Top20 图的对照可以得到两个在数值层面稳健的观察。第一,当模型整体性能较强(例如音频最优 WavLM 6TR layer9、以及多模态最优 Whisper-base 6TR layer2)时,Top20 ROI 的相关值分布整体上移;当模型整体性能较弱(例如文本 RoBERTa win200 layer4)时,Top20 ROI 的相关也明显较低。该现象与全脑均值趋势一致,提示性能差异并非由单一 ROI 的极端值主导,而更像是多个 ROI 上相关的整体抬升。第二,尽管强模型的 Top20 值整体更高,不同模型的 Top20 ROI 编号与排序并不完全一致,这意味着模型表征的优势可能集中在不同的 ROI 子集上。该差异为“模态偏好与语义偏好”的进一步分析提供了入口,但要把 ROI 编号解释为具体解剖区域或功能系统,需要额外的 ROI 命名映射文件与统计检验(例如多重比较控制)。这些内容在当前结果目录中尚未形成可追溯的输出,因此本报告在此不做超出文件所支持范围的机制性断言。

为了在同一页面中直观对照不同模态的 ROI Top20 分布,本报告将三张代表性 ROI 图并列展示如图 \ref{fig:roi_triplet}。该图的阅读方式是先比较数值范围与整体高度,再比较条目(ROI 编号)的重合程度,从而形成对“强模型是否带来更广泛的 ROI 提升”以及“不同模态是否偏向不同 ROI 子集”的直接感受。结合前述脑图(表面空间分布)与模型对比图(全脑均值趋势),ROI 图提供了从空间可视化走向区域定量对照的中间层证据。

\begin{figure}[H]
\centering
\begin{minipage}{0.32\linewidth}
\centering
\includegraphics[width=\linewidth]{roi_roberta_l4.png}
\caption*{文本:RoBERTa win200 layer4}
\end{minipage}\hfill
\begin{minipage}{0.32\linewidth}
\centering
\includegraphics[width=\linewidth]{roi_wavlm6tr_l9.png}
\caption*{音频:WavLM 6TR layer9}
\end{minipage}\hfill
\begin{minipage}{0.32\linewidth}
\centering
\includegraphics[width=\linewidth]{roi_whisper6tr_l2.png}
\caption*{多模态:Whisper-base 6TR layer2}
\end{minipage}
\caption{三种代表性配置的 ROI Top20 对照(由 report/scripts/make\_figures.py 从 results/roi.csv 生成)。}
\label{fig:roi_triplet}
\end{figure}
