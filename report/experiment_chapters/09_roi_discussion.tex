\section{时间尺度效应(TR 窗口)}
在音频与多模态结果中,TR 窗口增大会带来稳定的性能上升(图 \ref{fig:audio_tr_trend}、图 \ref{fig:mm_tr_trend}),融合结果也呈现同向趋势(图 \ref{fig:fusion_tr_trend})。该现象说明:在自然故事听觉范式下,TR 级输入的时间聚合尺度是影响编码性能的主导因素之一。以音频模型为例,WavLM 的最佳均值相关从 1TR 的 $0.0316$(layer9)上升到 6TR 的 $0.0916$(layer9);wav2vec2 从 1TR 的 $0.0274$(layer1)上升到 6TR 的 $0.0895$(layer9);HuBERT 从 1TR 的 $0.0326$(layer4)上升到 6TR 的 $0.0823$(layer9)。多模态 Whisper-base 的最佳均值相关从 1TR 的 $0.0270$(layer1)上升到 6TR 的 $0.0889$(layer2),CLAP 同样随窗口上升。窗口效应在不同模型家族中同时出现,表明性能差异的比较必须以相同 TR 窗口为前提,否则“窗口长度”本身会掩盖模型与层级差异。

\section{层级趋势(跨层对齐)}
为了更加准确的对比模型的层级差异,我们选择的模型全部是 12 层,其目的不是只报告“最优层”,而是用层级曲线回答“表征在模型内部的哪些抽象层级更容易与大脑对齐”。我们把同一评估协议下的层级结果按层汇总并可视化:文本侧为 win200(图 \ref{fig:text_layer_trend}),音频侧为 6TR(图 \ref{fig:audio_layer_trend})。

文本侧的层级曲线呈现明显的“模型依赖峰值层”:RoBERTa 的最优出现在中层(layer4,$0.0153 \pm 0.0160$),BERT 的最优出现在中高层(layer6,$0.0084 \pm 0.0161$),GPT-2 的最优出现在更高层(layer12,$0.0107 \pm 0.0137$)。同一“文本—fMRI”对齐任务上,不同语言模型把有效信息分配到不同层级位置,这使得跨模型比较必须同时考虑“模型差异”和“层级位置差异”,否则会把“峰值层位置变化”误当作“模型好坏变化”。 考虑到我们限制了模型的层数,所以关于文本模型层级趋势的结论仍然有限,如果纳入更多参数量级的模型(如 GPT-3、LLaMA 等),可能会观察到不同的层级趋势。

音频侧的层级曲线更集中:在 6TR 条件下,三种音频模型的峰值层都落在中高层(wav2vec2/WavLM/HuBERT 的最佳层分别为 layer9/layer9/layer9,对应 $0.0895$/$0.0916$/$0.0823$)。在较短窗口下,wav2vec2 的最佳层会更偏低层(例如 1TR/2TR 最佳层为 layer1),而窗口增大后峰值层移动到中高层(3TR/6TR 最佳层为 layer9)。层级曲线与窗口效应结合起来给出一个直接的结构性结论:在更长时间聚合尺度下,与 BOLD 更稳定对齐的表征更集中在“跨时间整合更强”的中高层输出,而不是仅反映局部输入变化的低层表示。

\InsertFig{text_layer_trend_win200.png}{0.92\linewidth}{文本模型:不同层的对齐性能(win200,均值$\pm$标准差,从 \texttt{results/summary.csv} 聚合)。}{fig:text_layer_trend}
\InsertFig{audio_layer_trend_6tr.png}{0.92\linewidth}{音频模型:不同层的对齐性能(6TR,均值$\pm$标准差,从 \texttt{results/summary.csv} 聚合)。}{fig:audio_layer_trend}

\section{模态差异(文本 vs 音频/多模态)}
在同一评估协议下,文本模型的整体相关量级显著低于音频与多模态:文本最佳约为 $0.015$(图 \ref{fig:text_best}),而音频与多模态在 6TR 条件下达到 $0.08\text{--}0.09$(图 \ref{fig:audio_tr_trend}、图 \ref{fig:mm_tr_trend})。这一差异与“特征对齐链路的压缩程度”一致:音频/多模态表征直接来自连续语音片段并在 TR 窗口上聚合,保留了与刺激同构的时间结构;文本表征来自转写序列,对齐到 TR 时必须把 token/词级序列在 TR 内再次聚合(本项目采用 TR 内平均池化),从而把 TR 内词数波动与边界效应折叠进单一均值向量。对齐链路上的信息压缩越强,进入回归的可用信号越弱,最终在相关指标上呈现量级差异。

另一条与模态差异一致的事实是:语音信号本身承载语言单位相关的结构(音素、音节、韵律、停顿与重音等),这些结构与词边界、句法分组和语义重点相关。因此强音频/语音表征不仅反映声学变化,也包含对语言结构的刻画;在听觉范式下,这种“声学+语言结构”的混合表征能够在 TR 聚合后形成更强、更稳定的预测信号,这为音频/多模态模型在总体量级上领先文本基线提供了直接解释链路。

\section{非线性对照(文本 aligned 特征)}
非线性读出器的对照结果与上述“特征主导”结论一致:核岭回归(RBF)的最优为 $0.0141 \pm 0.0186$(RoBERTa aligned\_layer1),低于线性文本基线的最优 $0.0153 \pm 0.0160$(RoBERTa layer4)。该对照在数值上体现为:即使把读出器从线性换成非线性形式,整体量级仍停留在 $10^{-2}$,并未跨越文本侧的瓶颈。

该现象与文本特征进入回归前的两次压缩过程一致:第一,token/词级序列表征在对齐到 fMRI 时被 TR 内平均池化为单个向量;第二,文本上下文窗口固定为 200 token,使跨 TR 的长程语义依赖在输入阶段被截断。压缩后的 TR 级文本特征在“时间结构”和“语义差异”两个维度上都更稀疏;当输入侧有效信息量不足时,读出器形式(线性/非线性)不会成为决定性因素,性能更取决于特征提取与对齐的质量与丰富度。

