\section{编码模型、评价指标与输出结构}
本文采用线性编码模型把刺激特征映射到 fMRI 响应。编码建模的选择不是因为线性模型能够穷尽语言加工的非线性机制,而是因为在自然故事范式下,线性正则化回归提供了一条可复现、可解释且便于跨模型比较的基线。语义地图研究以体素级(或 ROI 级)正则化线性回归在未见故事上预测 fMRI,并据此绘制语义选择性地图 \cite{Huth2016};集成建模工作进一步强调统一评估协议的重要性,使不同模型与不同层的差异更可能来自表征本身而非评估细节 \cite{Schrimpf2021}。在机制解释层面,Antonello 与 Huth 的讨论提示我们应避免把编码性能直接等同于某一种训练目标或单一认知机制,因此本文将编码性能视为一种“表征可预测性”指标,并通过多模型、多层与多窗口的系统比较减少偶然性 \cite{Antonello2023}。

\subsection{输入特征的标准化与降维}
对每一种特征,我们先对每一维做标准化,使其在样本维上的均值为零、方差为一。标准化的作用是避免不同特征维度的尺度差异影响岭回归的惩罚项,从而使正则化主要反映信息量而非尺度。随后,我们在特征维上执行主成分分析(PCA),默认降到 250 维。降维的动机有二:其一,预训练模型的隐藏维度通常较高且存在共线性,PCA 有助于稳定回归解;其二,后续时延展开会把特征维度按窗口倍数放大,若不降维会显著增加计算量并降低可重复性。融合实验在拼接后的联合特征空间上执行 PCA,从而使主成分同时反映文本与音频方差结构。

\subsection{FIR 时延展开与线性岭回归}
为处理 BOLD 信号相对刺激的延迟与时间平滑,我们对 TR 级特征做 FIR(finite impulse response)时延展开。设原始特征矩阵为 $X \in \mathbb{R}^{T \times D}$,我们构造延迟窗口长度为 $W$、偏移为 $O$ 的拼接特征
\[
\tilde{X}_t = [X_{t-O}, X_{t-O-1}, \ldots, X_{t-O-(W-1)}] \in \mathbb{R}^{W D},
\]
从而把“当前 BOLD”表示为过去若干个 TR 内刺激特征的线性组合。本文在已生成结果对应的默认设置中采用 $W=4$、$O=1$。在此基础上,我们对每个 ROI 分别拟合岭回归:
\[
\hat{Y} = \tilde{X} \beta,\quad \beta = \arg\min_{\beta} \|Y-\tilde{X}\beta\|_2^2 + \alpha \|\beta\|_2^2,
\]
其中 $Y \in \mathbb{R}^{T \times R}$ 为 ROI 级 fMRI 响应,$R$ 为 ROI 数。岭回归在高维共线特征下稳定且易于比较,是自然故事编码建模的常用选择。

\subsection{训练/测试划分、多被试汇总与指标定义}
本文的评估以“未见数据上的相关系数”为核心指标。对每个被试,我们先在时间轴两端各排除 10 个 TR,以降低边界效应。随后采用单次训练/测试划分:按时间顺序以前 80\% 的 TR 作为训练集、后 20\% 的 TR 作为测试集。该划分避免了 K 折交叉验证带来的显著计算开销,使多模型多层多窗口的系统比较在当前硬件条件下可行。为保证不同模型之间的可比性,我们在所有模型上使用相同的划分方式与相同的正则化系数设置,并在多被试上重复该评估。对每个 ROI,我们计算测试集预测值与真实值的皮尔逊相关系数,得到该被试的 corr map;再对 corr map 的 ROI 维做均值得到该被试的平均相关。最终报告的均值与标准差来自对所有被试平均相关的汇总。

\subsection{输出文件与可追溯性}
为保证论文结论可核验,本文所有结果均对应到仓库中已生成的文件。单模态与多模态(Whisper/CLAP)结果以 \texttt{results/GROUP/MODEL/SETTING/} 为目录结构,每个配置包含 \texttt{log.txt} 与若干 \texttt{corr\_layer*.npy},前者记录多被试均值与标准差,后者保存 ROI 级 corr map。融合结果保存到 \texttt{results/fusion/TEXT\_MODEL\_\_AUDIO\_MODEL/},每一组层与窗口配置保存为 \texttt{corr\_t*\_a*\_ctx*\_tr*.npy} 并在对应 \texttt{log.txt} 中记录统计量。统计图由 \texttt{report/scripts/make\_figures.py} 从 \texttt{results/summary.csv}、\texttt{results/roi.csv} 与融合日志聚合生成到 \texttt{report/figures/};皮层脑图由 \texttt{src/run\_plot\_corr\_maps.py} 将 \texttt{corr\_*.npy} 映射到皮层表面并输出到 \texttt{report/figures/brainmaps/}。这种“数值—可视化—文件路径”的对应关系保证了本文叙述可以被逐项复现与检查。
