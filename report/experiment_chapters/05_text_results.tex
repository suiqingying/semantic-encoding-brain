\section{实验结果(一):文本模型}
文本模型结果来自 \texttt{results/summary.csv} 中 \texttt{/text/} 条目及对应的 \texttt{corr\_layer*.npy}。本次文本特征采用固定 200 token 上下文窗口,并将词级表示在 TR 内取平均得到 TR 级特征。该设置在方法上提供了稳定对照,但也意味着文本表征受限于窗口长度与 TR 内词数波动。由于本文不引入超出已完成实验的推断,结果部分仅对当前设置下的数值与空间分布作出可追溯陈述。

三种文本模型的最佳层均值相关分别为:RoBERTa-base 在 layer4 达到 $0.0153 \pm 0.0160$,GPT-2 在 layer12 达到 $0.0107 \pm 0.0137$,BERT-base 在 layer6 达到 $0.0084 \pm 0.0161$。图 \ref{fig:text_best} 以柱状图形式对比了三种模型的最佳层表现。就数量级而言,文本模型在本任务中的可预测性显著低于音频与多模态模型(后文将给出 0.08--0.09 量级的强基线),因此文本结果在本文中主要承担“纯语义/文本基线”的作用,用于在相同评估协议下刻画语义特征在当前数据与时间尺度下的可预测上限。

值得注意的是,我们在使用 Transformer 类别的文本模型时,也有过比较好的结果,比如 Qwen2-0.5B 能够达到 0.0415 的均值相关。考虑到我们的实验目的在于比较不同模态的预训练模型在自然故事听觉范式下的语义编码能力,而非对单一模态模型进行极限性能挖掘,因此本文最终选择了 RoBERTa、GPT-2 与 BERT 三种经典且广泛使用的文本模型作为代表,来呈现文本模态在当前实验框架下的表现。

空间分布方面,图 \ref{fig:text_montage} 将三种文本模型的最佳层相关图并置,便于观察在同一绘图视角与同一色标下的差异。为避免把跨模型差异误读为绘图设置差异,本图的每一张脑图都来自相同的 corr map 到皮层表面映射流程。进一步地,图 \ref{fig:text_roberta_map}、图 \ref{fig:text_gpt2_map} 与图 \ref{fig:text_bert_map} 分别展示三种模型在其最佳层的单独脑图,从而便于在后续讨论 ROI 偏好与空间模式时引用。尽管文本模型整体相关较低,图 \ref{fig:text_roberta_roi} 的 ROI Top20 仍能为“哪些区域在当前框架下更容易被文本语义预测”提供定量入口,并可与音频/多模态的 ROI 分布形成对照。

\subsection{非线性补充:Kernel Ridge on aligned features}
除线性岭回归外,我们在文本对齐特征上补充评估了核岭回归(Kernel Ridge, RBF 核)。该非线性基线直接读取 \texttt{results/text/*/win200/aligned\_layer*.npy} 作为输入,在相同的 FIR 时延展开与 80\%/20\% 时间顺序划分下训练与评估,并把多被试统计写入 \texttt{results/nonlinear/log.txt}。在当前已生成的结果中,RoBERTa-base 的 aligned\_layer1 取得最高均值相关 $0.0141 \pm 0.0186$,GPT-2 的 aligned\_layer6 为 $0.0115 \pm 0.0154$,BERT-base-uncased 的结果整体更低且存在接近零或负值的配置(例如 aligned\_layer1 为 $-0.0007 \pm 0.0181$)。就最优均值而言,该非线性基线并未超过线性文本基线的最佳结果(RoBERTa-base layer4 的 $0.0153 \pm 0.0160$),提示在当前文本特征与样本规模下,简单非线性并不必然带来稳定增益;同时,由于该实现仅输出平均相关统计而不保存 corr map,本文不对其空间分布进行进一步比较。这其实也说明了,模型更加复杂并不一定带来更好的结果,特征本身的质量和适配性可能才是决定模型表现的关键因素。

\InsertFig{text_best.png}{0.86\linewidth}{文本模型最佳层的全脑均值相关系数(从 \texttt{results/summary.csv} 聚合)。}{fig:text_best}
\InsertFig{text_class_montage.png}{0.96\linewidth}{文本模型最佳层脑图对照:RoBERTa(win200,layer4)、GPT-2(win200,layer12)、BERT(win200,layer6)。}{fig:text_montage}
\InsertFig{text_roberta-base_win200_corr_layer4.png}{0.92\linewidth}{RoBERTa-base(win200,layer4)相关图可视化。}{fig:text_roberta_map}
\InsertFig{text_gpt2_win200_corr_layer12.png}{0.92\linewidth}{GPT-2(win200,layer12)相关图可视化。}{fig:text_gpt2_map}
\InsertFig{text_bert-base-uncased_win200_corr_layer6.png}{0.92\linewidth}{BERT-base(win200,layer6)相关图可视化。}{fig:text_bert_map}
\InsertFig{roi_roberta_l4.png}{0.86\linewidth}{RoBERTa-base(win200,layer4)对应相关图的 ROI Top20(从 \texttt{results/roi.csv} 聚合)。}{fig:text_roberta_roi}
