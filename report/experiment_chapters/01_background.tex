\chapter{背景、目标与本报告范围}
本项目采用自然故事听觉范式:被试连续听取长时语音刺激,研究者同时记录全脑 fMRI。与经典的离散刺激范式相比,这类数据的时间结构更接近真实语言理解过程,刺激在声学层面随时间快速变化,而语义与叙事层面的信息则跨句、跨段累积。对于编码建模而言,刺激的这种层级结构意味着两件事。第一,刺激特征必须被严格对齐到 fMRI 的采样时刻(TR),否则编码模型的输入输出不在同一时间轴上,任何“模型表征与脑表征的对齐”都会被时间错位掩盖。第二,BOLD 信号存在血氧动力学延迟与时间平滑,因此刺激特征需要在时间上做合适的聚合与延迟建模(例如 FIR 延迟拼接),以使线性模型能够在较低的复杂度下捕捉到主要的响应动力学。

本项目的研究目标是比较不同预训练模型、不同层的特征对于大脑反应的可预测性,并进一步分析不同脑区对不同模态特征的偏好。这里“可预测性”采用编码模型预测值与真实 fMRI 的相关系数作为度量;相关越高,表示该特征在当前编码框架下与脑信号对齐程度越高。本项目的实现强调可追溯性:每一次模型与层的评估都会在 \texttt{results/} 下产生对应的 \texttt{log.txt} 与 \texttt{corr\_layer*.npy} 文件,前者记录多被试的均值与标准差,后者保存每个 ROI 的相关图(corr map)。统计图由 \texttt{report/scripts/make\_figures.py} 直接读取 \texttt{results/summary.csv} 与 \texttt{results/roi.csv} 生成;脑图由 \texttt{src/run\_plot\_corr\_maps.py} 读取 \texttt{corr\_layer*.npy} 生成,并保存到 \texttt{report/figures/brainmaps/},从而保证“数值结果—可视化—原始文件路径”三者可以互相核验。

需要明确本报告的边界:本报告仅描述当前仓库中已生成并保存为文件的实验结果,不对尚未运行、运行失败、或未保存为结果文件的实验做任何陈述。尤其是非线性编码模型部分,在当前结果目录中未形成可用于对比的系统性输出,因此本报告只讨论线性编码与线性融合(特征拼接)在现有结果上的表现,并在讨论中说明未覆盖部分。
