\section{实验结果(四):文本--音频特征融合}
融合实验的目标是在不改变编码模型与评估协议的前提下,检验文本特征与音频特征是否在 TR 级时间轴上提供互补信息。与多模态模型“在预训练阶段引入跨模态对齐约束”不同,本文的融合采取更直接的特征级拼接:对同一 TR 的文本特征与音频特征分别标准化后在特征维上拼接,再在拼接后的联合空间执行 PCA,随后进入同一 FIR+岭回归评估流程。该实现对应 \texttt{results/fusion/} 目录,其中文本与音频模型对以子目录区分,具体配置以 \texttt{corr\_t*\_a*\_ctx*\_tr*.npy} 命名并在 \texttt{log.txt} 中记录统计量。

当前已生成的融合结果覆盖固定文本上下文窗口 $\texttt{ctx}=200$、三种音频 TR 窗口(1TR、2TR、3TR)、三种文本模型与三种音频模型,并在多层组合上形成可解析的配置集合。以融合日志解析得到的 540 条记录为基础,融合的全局最优配置出现在 $\texttt{tr}=3$:RoBERTa-base 的 $\texttt{text\_layer}=1$ 与 WavLM-base-plus 的 $\texttt{audio\_layer}=9$ 组合达到 $0.0535 \pm 0.0266$。分窗口看,$\texttt{tr}=1$ 的最优配置为 RoBERTa-base(layer1)+ HuBERT(layer4),均值相关为 $0.0330 \pm 0.0179$;$\texttt{tr}=2$ 的最优配置为 RoBERTa-base(layer1)+ WavLM(layer9),均值相关为 $0.0431 \pm 0.0212$;$\texttt{tr}=3$ 的最优配置即上述全局最优。图 \ref{fig:fusion_best_bar} 与图 \ref{fig:fusion_tr_trend} 分别展示融合 Top 配置的整体对比与窗口趋势,清晰呈现“窗口增大带来系统性提升”的规律。

融合结果的关键问题并不仅是“是否提高一个全局均值”,而是“文本层与音频层是否呈现交互结构”。若融合收益仅由某一侧单模态强信号驱动,则不同层组合应当在二维网格上近似单调;相反,若存在互补与匹配,则可能出现局部最优区域。图 \ref{fig:fusion_heatmap} 在固定 $\texttt{ctx}=200$、$\texttt{tr}=3$ 与全局最优模型对(RoBERTa + WavLM)条件下给出二维性能热图,可以看到性能分布并非简单随层号单调变化,而是在若干组合附近形成高值区域。这一现象与“层级位置决定表征抽象度与可用信息类型”的观点一致 \cite{Schrimpf2021,Antonello2023,Kumar2024},并为后续更系统的跨层融合策略提供了直接的经验约束。

空间分布方面,图 \ref{fig:fusion_montage} 展示融合 Top 配置的脑图对照,图 \ref{fig:fusion_best_map} 展示全局最优融合配置的单独脑图,从而把融合结果纳入与单模态一致的“统计图—ROI 图—脑图”证据链。需要强调的是,融合最优均值相关(约 0.053)仍明显低于音频与多模态在 6TR 条件下的 0.08--0.09 量级强基线,因此在当前结果覆盖范围内,特征拼接融合并未带来超越强音频模型的整体优势。该结论并不意味着跨模态互补不存在,而更可能反映当前融合只覆盖到 3TR 的时间尺度、以及联合 PCA 在有限样本下对跨模态方差结构的选择性等因素。对这些因素的分析将放在讨论部分展开。

\InsertFig{fusion_best.png}{0.96\linewidth}{融合:Top 配置的全脑均值相关对比(从 \texttt{results/fusion/**/log.txt} 解析聚合)。}{fig:fusion_best_bar}
\InsertFig{fusion_tr_window_trend.png}{0.86\linewidth}{融合:不同 TR 窗口下的全局最优性能趋势(从融合日志解析聚合)。}{fig:fusion_tr_trend}
\InsertFig{fusion_heatmap_bestpair.png}{0.78\linewidth}{融合热图:RoBERTa + WavLM 在 $\texttt{ctx}=200$、$\texttt{tr}=3$ 条件下的层交互结构(从融合日志解析生成)。}{fig:fusion_heatmap}
\InsertFig{fusion_montage.png}{0.96\linewidth}{融合脑图对照:融合 Top 配置的 corr map 脑图组合。}{fig:fusion_montage}
\InsertFig{fusion_roberta-base__microsoft_wavlm-base-plus_corr_t1_a9_ctx200_tr3.png}{0.92\linewidth}{全局最优融合配置脑图:RoBERTa-base(layer1)+ WavLM-base-plus(layer9),$\texttt{ctx}=200$,$\texttt{tr}=3$。}{fig:fusion_best_map}
