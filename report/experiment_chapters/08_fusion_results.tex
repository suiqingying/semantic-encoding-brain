\chapter{文本+音频融合结果:覆盖范围、最优配置与层交互结构}
融合实验的输出位于 \texttt{results/fusion/}。与单模态结果的 \texttt{corr\_layer*.npy} 命名不同,融合结果的文件名包含文本层、音频层、文本上下文窗口与音频 TR 窗口,例如 \texttt{corr\_t6\_a9\_ctx200\_tr3.npy}。融合的核心思想是把文本与音频在 TR 级对齐后进行特征拼接,再在同一套 PCA+FIR+岭回归框架下评估其对 fMRI 的可预测性。实现上,\texttt{src/run\_multimodal\_fusion.py} 对文本与音频分别做标准化(\texttt{StandardScaler}),将两者在维度上拼接得到融合特征,再在拼接后的联合空间执行 PCA(默认 250 维),最后做 FIR 延迟展开并回归到 360 个 ROI。由于 PCA 在联合空间上进行,主成分同时反映文本与音频的方差结构,因此“融合是否有效”不仅取决于单模态信息是否互补,也取决于联合空间中哪些方向更易被线性回归利用。

与此前仅有少量融合记录不同,当前目录中融合已经形成较大规模的可追溯输出:在 \texttt{ctx=200} 的固定设置下,融合覆盖 \texttt{tr\_win=1/2/3} 三种窗口、3 个文本模型与 3 个音频模型,以及多层组合,最终在 \texttt{results/fusion/} 下生成大量 \texttt{corr\_t*\_a*\_ctx*\_tr*.npy} 文件并写入对应的 \texttt{log.txt}。由于实际可运行的层集合受到“是否存在对应特征文件”的约束,融合并非严格的全笛卡尔积;但在当前结果中,融合日志已经包含 540 条可解析的配置记录,这使得我们可以直接在融合内部比较不同 TR 窗口、不同模型对与不同层组合的影响。

图 \ref{fig:fusion_best_bar} 给出融合中均值相关最高的 Top12 配置,图 \ref{fig:fusion_tr_trend} 给出融合在不同 TR 窗口下的全局最优趋势。两张图共同表明:融合最优值随着 TR 窗口增大而显著上升,当前全局最优出现在 \texttt{tr\_win=3}。基于融合日志的扫描,融合的全局最优配置为 \texttt{bert-base-uncased} 与 \texttt{microsoft/wavlm-base-plus} 的组合,其均值相关为 0.0535(标准差 0.0216),对应 \texttt{text\_layer=6}、\texttt{audio\_layer=9}、\texttt{ctx\_words=200}、\texttt{tr\_win=3}。该数值仍低于音频与多模态在 6TR 条件下的 0.08--0.09 量级结果,因此融合是否能在更长窗口下进一步接近或超过强音频基线,需要在后续补齐 \texttt{tr\_win=6} 与更完整特征覆盖后才能回答。本报告在此仅对当前已生成的融合结果进行严格陈述与可视化总结。

融合的重要问题不是“是否只提升一个数值”,而是“文本层与音频层是否存在交互”。为此,图 \ref{fig:fusion_heatmap} 以全局最优模型对为例,在固定 \texttt{ctx=200, tr=3} 的条件下绘制 \texttt{text\_layer} 与 \texttt{audio\_layer} 的二维性能热图。该图在数值上展示了一个稳定事实:不同层组合的性能并非单调随层数增加而提升,而是存在若干局部最优区域,提示融合效果依赖于两种表征的“相对抽象层级匹配”,而非简单地拼接任意深层即可获益。

空间层面上,图 \ref{fig:fusion_montage} 展示融合 Top6 配置的脑图对照,图 \ref{fig:fusion_best_map} 展示全局最优融合配置的单独脑图。两类图像的作用是把融合结果纳入与单模态一致的“统计图—ROI 图—脑图”证据链,使得后续可以在相同视角下对比融合与单模态的空间分布差异,并检查融合是否在部分 ROI 上更接近文本或音频的模式。

\InsertFig{fusion_best.png}{0.96\linewidth}{融合:Top12 配置的全脑均值相关对比(由 report/scripts/make\_figures.py 从 results/fusion/**/log.txt 解析生成)。}{fig:fusion_best_bar}
\InsertFig{fusion_tr_window_trend.png}{0.86\linewidth}{融合:不同 TR 窗口下的全局最优性能趋势(由 report/scripts/make\_figures.py 从融合日志解析生成)。}{fig:fusion_tr_trend}
\InsertFig{fusion_heatmap_bestpair.png}{0.78\linewidth}{融合热图:全局最优模型对(BERT + WavLM)在 ctx=200、tr=3 条件下的层交互结构(由 report/scripts/make\_figures.py 解析融合日志并绘制)。}{fig:fusion_heatmap}
\InsertFig{fusion_montage.png}{0.96\linewidth}{融合脑图对照:融合 Top6 配置的 corr map 脑图组合(由 src/run\_plot\_corr\_maps.py 从 results/fusion 的 corr\_*.npy 选取并绘制)。}{fig:fusion_montage}
\InsertFig{fusion_bert-base-uncased__microsoft_wavlm-base-plus_corr_t6_a9_ctx200_tr3.png}{0.92\linewidth}{全局最优融合配置脑图:bert-base-uncased(layer6)+ WavLM-base-plus(layer9),ctx=200,tr=3(由 src/run\_plot\_corr\_maps.py 从对应 corr\_*.npy 绘制)。}{fig:fusion_best_map}
