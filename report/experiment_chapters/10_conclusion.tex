\section{结论、局限与后续可扩展方向}
在当前已经完成并保存为结果文件的实验范围内,编码模型的可预测性呈现出清晰且可复现的模态差异。文本模型在 \texttt{win200} 的设置下整体相关较低,即便在各自最佳层,其跨被试均值相关仍处于 $10^{-2}$ 量级;音频模型在较长 TR 窗口下达到约 0.08--0.09 的均值相关,并且不同音频模型在窗口增大时都呈现一致的性能提升趋势;多模态模型中 Whisper-base 在 6TR 条件下取得与强音频基线接近的性能,CLAP 在当前完成范围内略低但同样随窗口增长而提升。上述结论既体现在 \texttt{results/summary.csv} 的均值对比上,也在脑图与 ROI Top20 图上得到一致支持:强模型不仅抬升全脑均值,也使多个 ROI 的相关分布整体上移。

融合实验已经形成大规模可追溯输出。与此前仅能展示少量融合示例不同,当前融合在 \texttt{ctx=200} 下覆盖 \texttt{tr\_win=1/2/3} 三种窗口、3 个文本模型与 3 个音频模型以及多层组合,融合日志可解析记录达到 540 条。融合的全局最优出现在 \texttt{tr\_win=3},对应 \texttt{bert-base-uncased + microsoft/wavlm-base-plus} 的层组合(\texttt{t6\_a9\_ctx200\_tr3}),跨被试均值相关达到 0.0535。该数值显著高于文本单模态并在融合内部呈现清晰的窗口效应与层交互结构,但仍低于音频与多模态在 6TR 条件下的 0.08--0.09 量级结果。因此,对“融合是否优于强音频基线”的最终回答仍取决于是否能在更长窗口(例如 6TR)与更一致的特征覆盖条件下完成对等比较。

本报告的第二个重要结论不是某个单一数值,而是本项目的可追溯链路已经被建立并可稳定复用。每一条结果都能从报告中的统计图或脑图回溯到唯一的源文件路径,进一步回溯到生成该文件的脚本与配置,从而使后续扩展实验(例如增加更多文本模型、补齐多模态模型、系统搜索窗口与池化策略、或加入显著性检验)可以在不破坏现有结构的前提下增量进行。与此同时,本报告也明确存在当前结果目录所决定的局限:其一,回归超参与划分策略在当前配置下未进行交叉验证选择,因此结果更适合用于特征对比的基线,而不适合用于对绝对性能做过度外推;其二,ROI 编号尚未映射到解剖名称,限制了对“语义偏好性”的命名解释;其三,非线性编码模型在当前结果目录中未形成与线性结果同结构的 corr map 输出,因此无法纳入同一套图像证据链进行比较。

在上述边界内,本报告已经完成了对现有结果的系统整理:对多模型、多层、多窗口的线性编码结果给出数值对比;对文本、音频、多模态三个类别分别提供“类别脑图对照”与“最优配置脑图”;在 ROI 层面展示代表性模型的 Top20 分布并讨论其可解释性与局限。后续工作的关键并不是对文字叙述做任何“补写”,而是在现有流水线中补齐缺失的实验维度,使这些章节能够在同一模板下自然扩展并与综述合并。
