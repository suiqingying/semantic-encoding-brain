% !TeX program = xelatex
% !TeX output-file = experiment_report.pdf
\documentclass[UTF8,a4paper,12pt,fontset=none]{ctexart}

% ===================== 1. 基础页面设置 =====================
\usepackage[a4paper,
  top=2.8cm, bottom=2.8cm, left=2.4cm, right=2.4cm,
  headheight=20pt, marginparwidth=2cm
]{geometry}
\usepackage{setspace}
\setstretch{1.45}
\usepackage{indentfirst}
\setlength{\parindent}{2em}
\usepackage{microtype}
\usepackage{enumitem}
\setlist{nosep}
\usepackage{tabularx}
\usepackage{amsmath}
\usepackage{amssymb}
\usepackage{booktabs}
\usepackage{longtable}
\usepackage{float}

% ===================== 2. 字体设置 =====================
\usepackage{fontspec}
\newcommand{\LatinMainPref}{TeX Gyre Pagella}
\newcommand{\LatinSansPref}{Noto Sans}
\newcommand{\LatinMonoPref}{JetBrains Mono}
\newcommand{\CJKMainPref}{Noto Serif CJK SC}
\newcommand{\CJKSansPref}{Noto Sans CJK SC}
\IfFontExistsTF{\LatinMainPref}{\setmainfont{\LatinMainPref}}{\setmainfont{Latin Modern Roman}}
\IfFontExistsTF{\LatinSansPref}{\setsansfont{\LatinSansPref}}{\setsansfont{Latin Modern Sans}}
\IfFontExistsTF{\LatinMonoPref}{\setmonofont{\LatinMonoPref}}{\setmonofont{Latin Modern Mono}}
\IfFontExistsTF{\CJKMainPref}{\setCJKmainfont{\CJKMainPref}}{\setCJKmainfont{FandolSong-Regular}}
\IfFontExistsTF{\CJKSansPref}{\setCJKsansfont{\CJKSansPref}}{\setCJKsansfont{FandolHei-Regular}}
\setCJKmonofont{\CJKSansPref}
\IfFontExistsTF{\CJKSansPref}{
  \newCJKfontfamily{\TitleCJK}{\CJKSansPref}[FakeBold=2]
}{
  \newCJKfontfamily{\TitleCJK}{FandolHei-Regular}[FakeBold=2]
}

% ===================== 3. 高级配色方案 =====================
\usepackage{xcolor}
\definecolor{BrandColor}{HTML}{005F6B}
\definecolor{AccentColor}{HTML}{D97757}
\definecolor{DarkText}{HTML}{1A1C20}
\definecolor{LightGrayBG}{HTML}{F4F6F7}
\definecolor{SectionNum}{HTML}{8BA6AC}
\colorlet{Primary}{BrandColor}
\colorlet{Secondary}{BrandColor!85!black}
\colorlet{Ink}{DarkText}

% ===================== 4. 图形与视觉组件 =====================
\usepackage{graphicx}
\usepackage{caption}
\captionsetup{font={small,sf,color=Ink!80},labelfont={bf,color=BrandColor},labelsep=quad}
% 编译工作目录为仓库根目录,因此这里使用 report 下的相对路径
\graphicspath{{report/figures/}{report/figures/brainmaps/}}
\usepackage{tikz}
\usetikzlibrary{shapes, shadows, positioning, calc, backgrounds, fadings}
\usepackage{tcolorbox}
\tcbuselibrary{skins, breakable}

\newtcolorbox{AbstractBox}{
  enhanced,
  breakable,
  colback=BrandColor!4,
  colframe=BrandColor,
  coltitle=BrandColor,
  fonttitle=\TitleCJK\bfseries\large,
  title={摘\quad 要},
  detach title,
  before upper={\tcbtitle\par\vspace{0.5em}},
  frame hidden,
  leftrule=3pt,
  arc=0mm,
  left=15pt, right=10pt, top=10pt, bottom=10pt,
  shadow={0mm}{0mm}{0mm}{white}
}

% ===================== 5. 标题样式系统 =====================
\usepackage{titlesec}
\titleformat{\section}
  {\TitleCJK\sffamily\bfseries\huge\color{BrandColor}}
  {\makebox[0pt][l]{\hspace{-0.4em}\raisebox{1.5ex}{\color{SectionNum!25}\fontsize{70}{70}\selectfont\thesection}}}
  {0em}
  {}
  [{\vspace{-0.5ex}\color{BrandColor!30}\titlerule[2pt]}]
\titlespacing*{\section}{0pt}{4.5ex plus 1ex minus .2ex}{2.5ex plus .2ex}

\titleformat{\subsection}
  {\TitleCJK\sffamily\Large\bfseries\color{DarkText}}
  {}
  {0em}
  {\tikz[baseline=(current bounding box.east),outer sep=0pt]
    \fill[AccentColor] (0,0) rectangle (0.6em, 1.2em);
   \hspace{0.6em}\thesubsection\hspace{0.8em}}
\titleformat{name=\subsection,numberless}
  {\TitleCJK\sffamily\Large\bfseries\color{DarkText}}
  {}
  {0em}
  {\tikz[baseline=(current bounding box.east),outer sep=0pt]
    \fill[AccentColor] (0,0) rectangle (0.6em, 1.2em);
   \hspace{0.6em}}
\titlespacing*{\subsection}{0pt}{3ex plus 1ex minus .2ex}{1.5ex plus .2ex}

\titleformat{\subsubsection}
  {\TitleCJK\sffamily\large\bfseries\color{BrandColor}}
  {\thesubsubsection}
  {1em}
  {}

% ===================== 6. 目录重构 =====================
\usepackage{titletoc}
\renewcommand{\contentsname}{\TitleCJK\bfseries\Huge 目录}
\titlecontents{section}[2.5em]
  {\addvspace{12pt}\sffamily\bfseries\large}
  {\contentslabel[\color{BrandColor}\thecontentslabel]{2.5em}}
  {}
  {\hfill\color{BrandColor}\contentspage}
\titlecontents{subsection}[5.5em]
  {\addvspace{3pt}\sffamily\small}
  {\contentslabel[\color{Ink!60}\thecontentslabel]{3.0em}}
  {}
  {\titlerule*[0.8pc]{.}\contentspage}

% ===================== 7. 页眉页脚 =====================
\usepackage{fancyhdr}
\pagestyle{fancy}
\setlength{\headheight}{24pt}
\fancyhf{}
\fancyhead[L]{\sffamily\color{Ink!40}\footnotesize \leftmark}
\fancyhead[R]{\sffamily\color{Ink!40}\footnotesize \CourseName}
\fancyfoot[C]{
  \tikz\node[
    rectangle,
    fill=BrandColor,
    text=white,
    rounded corners=2pt,
    inner sep=3pt,
    minimum width=2em
  ]{\small\thepage};
}
\renewcommand{\headrulewidth}{0pt}
\renewcommand{\footrulewidth}{0pt}

% ===================== 8. 链接设置 =====================
\usepackage{hyperref}

% ===================== 信息宏定义 =====================
\newcommand{\StudentName}{潘宇轩}
\newcommand{\StudentID}{2023K8009991004}
\newcommand{\Department}{人工智能学院}
\newcommand{\Major}{人工智能}
\newcommand{\CourseName}{认知神经科学}

\newcommand{\PaperTitle}{语义编码与多模态对齐:实验报告}
\newcommand{\PaperSubtitle}{基于故事听觉 fMRI 的预训练模型特征比较}
\newcommand{\PaperTitleEn}{Semantic encoding and multimodal alignment: an experimental report}

\hypersetup{
  colorlinks=true,
  linkcolor=BrandColor,
  urlcolor=AccentColor,
  citecolor=BrandColor,
  pdfborder={0 0 0},
  pdftitle={\PaperTitle},
  pdfauthor={\StudentName}
}

% ===================== 兼容性层 (Shim) =====================
\newcommand{\frontmatter}{}
\newcommand{\mainmatter}{}
\newcommand{\backmatter}{}
\let\realSection\section
\let\realSubsection\subsection
\let\realSubsubsection\subsubsection
\makeatletter
\NewDocumentCommand{\chapter}{s o m}{%
  \IfBooleanTF{#1}{%
    \IfNoValueTF{#2}{\realSection*{#3}}{\realSection*[#2]{#3}}%
  }{%
    \IfNoValueTF{#2}{\realSection{#3}}{\realSection[#2]{#3}}%
  }%
}
\RenewDocumentCommand{\section}{s o m}{%
  \IfBooleanTF{#1}{%
    \IfNoValueTF{#2}{\realSubsection*{#3}}{\realSubsection*[#2]{#3}}%
  }{%
    \IfNoValueTF{#2}{\realSubsection{#3}}{\realSubsection[#2]{#3}}%
  }%
}
\RenewDocumentCommand{\subsection}{s o m}{%
  \IfBooleanTF{#1}{%
    \IfNoValueTF{#2}{\realSubsubsection*{#3}}{\realSubsubsection*[#2]{#3}}%
  }{%
    \IfNoValueTF{#2}{\realSubsubsection{#3}}{\realSubsubsection[#2]{#3}}%
  }%
}
\makeatother

% ===================== 全局图片插图命令 =====================
\newcommand{\InsertFig}[4]{%
  \begin{figure}[H]
    \centering
    \includegraphics[width=#2]{#1}%
    \caption{#3}%
    \label{#4}%
  \end{figure}%
}

% ===================== 附录:原始文件嵌入 =====================
\usepackage{fancyvrb}

\begin{document}

% ===================== 封面设计(复用综述风格) =====================
\begin{titlepage}
  \thispagestyle{empty}
  \newgeometry{margin=0cm}
  \begin{tikzpicture}[remember picture,overlay]
    \path[shade, top color=BrandColor!88!black, bottom color=BrandColor!8]
      (current page.north west) rectangle (current page.south east);
    \foreach \x in {0.08,0.20,...,0.92} {
      \draw[white, opacity=0.04, line width=0.6pt]
        ([xshift=\x\paperwidth]current page.north west) -- ([xshift=\x\paperwidth]current page.south west);
    }
    \node[
      anchor=center,
      yshift=0.02\paperheight,
      rounded corners=4mm,
      fill=white,
      fill opacity=0.86,
      text opacity=1,
      draw=white,
      draw opacity=0.15,
      line width=0.6pt,
      inner xsep=18pt,
      inner ysep=16pt,
      drop shadow={opacity=0.12, shadow xshift=0mm, shadow yshift=-1mm}
    ] at (current page.center) {%
      \begin{minipage}{0.78\paperwidth}
        \centering
        \tikz\node[fill=BrandColor!10, text=BrandColor, rounded corners=20pt, inner sep=9pt]
          {\sffamily\bfseries\large \CourseName};\\[1.0cm]

        {\rmfamily\bfseries\fontsize{33}{44}\selectfont\color{DarkText} \PaperTitle}\\[0.35cm]
        {\TitleCJK\sffamily\Large\color{DarkText!80} \PaperSubtitle}\\[0.45cm]
        {\fontspec{TeX Gyre Pagella}\itshape\fontsize{13}{18}\selectfont\color{Ink!65} \PaperTitleEn}\\[0.9cm]

        \tikz\draw[AccentColor, line width=1.4pt] (0,0) -- (4,0);\\[0.9cm]

        \renewcommand{\arraystretch}{1.55}
        \sffamily\color{DarkText}
        \begin{tabular}{r@{\hspace{1em}}l}
          \textbf{\color{BrandColor!80} 报告人} & \large \StudentName \\
          \textbf{\color{BrandColor!80} 学\quad 号} & \large \StudentID \\
          \textbf{\color{BrandColor!80} 院\quad 系} & \large \Department \\
          \textbf{\color{BrandColor!80} 专\quad 业} & \large \Major \\
          \textbf{\color{BrandColor!80} 日\quad 期} & \large \today \\
        \end{tabular}
      \end{minipage}%
    };
  \end{tikzpicture}
\end{titlepage}
\restoregeometry

% ===================== 摘要页(不使用列表,全文段落) =====================
\clearpage
\begin{AbstractBox}
本实验报告基于当前项目目录中已生成的结果文件撰写,所有数值均来自 \texttt{results/summary.csv}、\texttt{results/roi.csv} 与各模型目录下保存的相关图(\texttt{corr\_layer*.npy} 以及融合的 \texttt{corr\_*.npy})。研究目标是在故事听觉 fMRI 数据上,对比文本模型、音频模型、多模态模型在不同层与不同时间窗口(TR 窗口)条件下的脑预测性能,并结合 ROI 统计与可视化图像分析区域偏好。实验采用统一的对齐、降维与时延展开流程,并在多被试上进行单次训练/测试划分的岭回归评估,从而得到可复现的全脑相关图。报告呈现两类图像:其一为统计图(模型对比与窗口趋势、ROI Top20),由 \texttt{report/scripts/make\_figures.py} 读取现有结果自动生成;其二为脑图,可视化来自 \texttt{src/run\_plot\_corr\_maps.py} 对保存的 corr map 进行映射绘制。最终结论以当前“已经完成”的实验为依据:音频表征在长窗口下显著优于文本表征;Whisper 的多模态表征与强音频基线接近;文本+音频融合目前仅完成部分组合与窗口,尚未显示超过强音频基线的优势。
\end{AbstractBox}
\clearpage

% ===================== 目录页 =====================
\clearpage
{
  \newgeometry{top=3cm, bottom=3cm, left=3cm, right=3cm}
  \hypersetup{linkcolor=DarkText}
  \tableofcontents
  \restoregeometry
}
\clearpage

% ===================== 正文(章节拆分在独立文件中) =====================
\section{背景与研究问题}
\begin{tcolorbox}[
  enhanced,
  breakable,
  colback=Nord6,
  colframe=Nord8,
  frame hidden,
  leftrule=4pt,
  arc=2mm,
  left=12pt,right=10pt,top=10pt,bottom=10pt
]
{\TitleCJK\bfseries\Large\color{Nord10}小组成员与分工}\par
\vspace{0.6em}
{\renewcommand{\arraystretch}{1.25}%
\begin{tabularx}{\linewidth}{@{}p{0.34\linewidth}X@{}}
\MemberInfo{\StudentName}{\StudentID}{panyuxuan231@mails.ucas.ac.cn} & \vspace{0pt}负责整体方案设计与实现;完成文本/音频/多模态与融合特征提取、线性编码模型与评估、ROI 分析与可视化;整理结果并撰写报告。\\[0.6em]
\MemberInfo{林俊杰}{N/A}{N/A} & \vspace{0pt}未参与本次实验执行与代码实现。\\[0.6em]
\MemberInfo{王煜}{N/A}{N/A} & \vspace{0pt}未参与本次实验执行与代码实现。\\[0.6em]
\MemberInfo{李向阳}{N/A}{N/A} & \vspace{0pt}未参与本次实验执行与代码实现。\\[0.6em]
\MemberInfo{朱奕}{N/A}{N/A} & \vspace{0pt}未参与本次实验执行与代码实现。\\[0.6em]
\end{tabularx}}
\end{tcolorbox}

本项目研究的问题是:在自然故事听觉范式下,预训练模型得到的刺激表征在多大程度上能够预测全脑 fMRI 响应,并且这种“对齐程度”在不同模态(文本、音频、多模态)与不同模型层级之间如何变化。该问题的出发点来自两条相互推进的研究线索。第一条线索是自然语音刺激下的语义系统映射。Huth 等使用长时自然故事并构建体素级编码模型,证明在严格的未见故事评估下,线性回归可以得到稳定可复现的语义地图,从而将“可预测性”作为语义表征的可测证据 \cite{Huth2016}。Zhang 等进一步把分析单位从语义类别扩展到语义关系,指出概念与关系并非由解剖上隔离的模块分别承载,而是通过跨网络的重叠模式编码并支持语义推理 \cite{Zhang2020}。第二条线索来自预训练模型与大脑对齐的集成建模。Schrimpf 等在统一评估协议下系统比较不同架构与不同层,发现 Transformer 模型通常更能解释语言相关脑区的活动,并强调跨模型的可比性与评估一致性 \cite{Schrimpf2021}。然而,“语言模型拟合脑数据”并不自动推出“大脑在执行下一词预测”。Antonello 与 Huth 通过多项分析指出,模型对齐可以由更一般的特征发现与语言结构归纳解释,且模型内部最擅长预测未来词的层并不必然是最佳脑编码层 \cite{Antonello2023}。因此,本项目把“多模型、多层”作为基本比较单位,以避免仅凭单一指标外推机制结论。

自然故事听觉范式的另一个关键维度是时间尺度。BOLD 信号相对刺激存在血氧动力学延迟与时间平滑,因此对齐必须同时解决“刺激与 TR 的时间对应”和“刺激对 BOLD 的时延影响”。语义地图研究通常通过把词级语义特征聚合到 TR,并在回归输入端引入时延展开来吸收动力学差异 \cite{Huth2016}。在本项目中,我们把这一逻辑推广到预训练模型特征:文本侧以 token 上下文窗口构造词级表示,再聚合到 TR;音频侧以 TR 窗口切分语音波形,把短时帧级隐藏状态池化为每个 TR 的表征,再进入同一回归框架。由于时间尺度会影响模型能否捕获长程语境,本项目还系统比较不同 TR 窗口长度对音频与多模态表征的影响,并在融合实验中检验跨模态信息是否互补。

在模型选择上,本项目覆盖三类表征。文本模型选用 GPT-2、BERT 与 RoBERTa,分别代表自回归 Transformer、双向 Transformer 与改进的掩码建模框架;这些模型作为上下文表征的代表,在以往语言脑对齐研究中常被用于区分不同预训练目标对表征结构的影响 \cite{Peters2018,Devlin2019,Radford2019}。音频模型选用 wav2vec2、WavLM 与 HuBERT,它们属于以波形为输入的自监督/弱监督语音表征家族,其中 wav2vec 2.0 的掩码预测目标为“从上下文恢复被遮蔽的离散语音单元”提供了具体实现 \cite{Baevski2020};相关研究表明这类模型在层级上呈现从声学到更抽象结构的渐变,并能在一定程度上对齐皮层的语音处理层级 \cite{Millet2023}。多模态模型部分覆盖 Whisper 与 CLAP 等模型的可用输出,用于检验“多模态训练或共享嵌入空间”是否能带来超越强音频基线的可预测性,并观察其空间分布是否更接近语义系统。

在研究问题的表述上,本文不把对齐结果直接解释为某一种认知机制的证据,而是以可复现的证据链回答三个可检验的问题。第一,在统一的编码评估框架下,不同模态与不同模型的对齐强弱排序如何,哪些配置构成强基线。第二,在模型内部的层级结构上,最佳层是否稳定出现在中间层或深层,以及这种层级位置是否与时间窗口长度共同作用。第三,当文本与音频特征进行简单拼接融合时,性能是否出现稳定提升,并且最佳文本层与最佳音频层是否呈现非单调交互。讨论部分将在这些结果约束下结合组合语义与 supra-word 表征观点 \cite{Toneva2022}、Transformer 功能分化分析 \cite{Kumar2024}、以及语义重构方向的互补视角 \cite{Tang2023},对结果的意义、不足与后续扩展方向作出解释。

\section{数据、对齐表与 TR 级刺激构建}
本项目使用的原始文件位于 \texttt{data/raw/}。fMRI 数据以 ROI 形式预先整理为 \texttt{21styear\_all\_subs\_rois.npy},对齐表位于 \texttt{21styear\_align.csv},音频刺激位于 \texttt{21styear\_audio.wav}。\texttt{src/data.py} 将这些文件加载为可被特征抽取与编码建模直接使用的结构:\texttt{load\_fmri()} 返回一个以被试编号为键的字典,每个条目是形状为 $(T,360)$ 的矩阵,表示 $T$ 个 TR 上 360 个 ROI 的 fMRI 信号;\texttt{load\_audio()} 以固定采样率读取整段音频;\texttt{load\_align\_df()} 读取对齐表并为每个词构造 TR 编号。

对齐表 \texttt{21styear\_align.csv} 每行包含四列:保留大小写的词、全部小写的词、词开始时间戳(秒)与词结束时间戳(秒)。对齐表中存在缺失项,代码对时间戳进行向后填充,并将缺失词以 \texttt{none} 作为占位。随后根据 TR 时长将词级时间戳映射到离散 TR 索引。项目设置 TR 为 1.5 秒,因此对于词开始时间 $t$(单位秒),对应 TR 索引为 $\lceil t/\mathrm{TR}\rceil$。这一步的输出是一个包含 \texttt{tr} 列的数据框,它把每个词归入某一个 TR,从而为后续“把词级特征聚合为 TR 级特征”提供了确定的分组键。

TR 级刺激构建需要同时处理三条时间轴:词级时间轴(用于文本与文本端对齐)、连续波形时间轴(用于音频分片)、TR 采样时间轴(用于与 fMRI 对齐),以及 BOLD 延迟轴(用于 FIR 延迟展开)。本项目对文本与音频采取统一的策略:先在原始粒度上抽取预训练模型特征,再将特征聚合到 TR。对于文本而言,\texttt{src/text\_pipeline.py} 先为每个词构造上下文窗口(默认 200 token),输入语言模型得到词级或 token 级表征,随后按 \texttt{tr} 分组,对同一 TR 中所有词的表征求均值得到 TR 级文本特征。运行时可能出现 pandas 的 FutureWarning,这属于 API 行为变更提示,不影响当前版本下的数值计算与输出文件。

对于音频而言,连续波形根据 TR 窗口切分为一系列 chunk。配置 \texttt{src/config.py} 中的 \texttt{AUDIO\_SR=16000} 表示采样率为 16kHz;当窗口设置为 1TR、2TR、3TR、6TR 时,分别对应 1.5s、3.0s、4.5s、9.0s 的音频片段。每个片段作为一个输入样本送入音频模型得到表示,形成与 TR 一一对应的序列。窗口长度不仅决定了音频表征是否覆盖跨 TR 的韵律与语音单位结构,也会与 FIR 延迟展开共同决定“刺激历史覆盖范围”,因此音频模型部分会系统比较不同 TR 窗口的效果。

为了使后续编码模型稳定训练,特征在进入回归之前会进行降维。当前实现默认使用 PCA 将 TR 级特征降到 250 维(\texttt{DEFAULT\_PCA\_DIM=250}),其目的在于减轻高维特征与有限样本量组合导致的病态问题,并降低回归求解成本。所有预处理都在特征与脑信号完成 TR 级对齐之后进行,从而保证特征矩阵与 fMRI 的时间轴严格一致。

\chapter{预训练模型、层选择策略与特征提取实现}
\subsection{模型集合与本次报告覆盖范围}
本项目将刺激表示划分为三类:文本模型表示、音频模型表示与多模态模型表示。文本模型部分在当前结果中覆盖 \texttt{gpt2}、\texttt{bert-base-uncased} 与 \texttt{roberta-base};音频模型部分覆盖 \texttt{facebook/wav2vec2-base-960h}、\texttt{microsoft/wavlm-base-plus} 与 \texttt{facebook/hubert-base-ls960};多模态模型部分覆盖 Whisper(\texttt{openai/whisper-small}、\texttt{openai/whisper-base})与 CLAP(\texttt{laion/clap-htsat-unfused})。以上模型的可比较结果体现在 \texttt{results/summary.csv} 中,并且每一条统计记录都可以追溯到对应的 \texttt{results/.../log.txt} 与 \texttt{corr\_layer*.npy} 文件。

在方法论上,本项目遵循“以预训练模型的多层隐藏状态作为可解释特征空间”的通行做法。语言模型方面,ELMo 的深层上下文表示 \cite{Peters2018}、BERT 的双向 Transformer 表示 \cite{Devlin2019} 与 GPT-2 的自回归表示 \cite{Radford2019} 构成了常用对照组,用以区分不同训练目标与不同上下文利用方式对脑预测性能的影响。语音模型方面,自监督框架 wav2vec 2.0 \cite{Baevski2020} 提供了从原始波形到高层语音表征的分层表示,并被用于检验模型对齐是否更接近真实语音加工通路 \cite{Millet2023}。在更宏观的模型比较工作中,大规模集成建模强调“层选择”对脑拟合度的关键作用 \cite{Schrimpf2021,Kumar2024},同时也提醒我们,模型对齐的来源可能既包含预测目标,也包含更一般的特征发现与迁移能力 \cite{Antonello2023}。

\subsection{层选择:按相对深度等比例取样}
不同预训练模型的层数并不相同,例如多数 base 级 Transformer 编码器为 12 层,而 Whisper-base 的编码器层数更少。如果直接固定使用某些绝对层号(例如一律抽取第 12 层),会导致在浅层模型中越界,或在深层模型中取样过稀,从而让“层对齐差异”混入“层号不匹配”带来的偏差。本项目的层选择采用等比例的相对策略:对每个模型先从配置中读取总层数 $L$,再用等间距取样从 1 到 $L$ 选取若干层,并四舍五入去重得到最终层集合。这一策略的直接结果是:对于 12 层模型,典型层集合为 $\{1,4,6,9,12\}$;对于 6 层模型,则可能得到 $\{1,2,4,5,6\}$。因此,本报告中“layer=k”的含义始终是“该模型结构中的第 $k$ 层”,而不是跨模型共享的绝对语义层级;跨模型比较时,我们把其理解为“从浅到深的相对位置”,并结合每个模型的层数解释其表现。

\subsection{文本特征:上下文窗口与双层池化}
文本特征提取遵循“词级上下文—词级表征—TR 级聚合”的流程。\texttt{src/run\_text\_models.py} 以对齐表为索引,为每个词构造长度为 200 token 的上下文窗口(\texttt{ctx\_words=200}),并将“预分词后的词序列”输入 HuggingFace 模型得到隐藏层输出。代码层面存在两次池化:第一次发生在模型输出端,用于将 token 序列压缩为一个词窗口的向量,支持最后 token 表征或 token 平均;第二次发生在时间对齐阶段,即将同一 TR 内所有词的向量做平均得到 TR 级表示。当前结果文件对应的实现采用“模型端取 last token 表征,TR 内对词向量做平均”的组合,这与自然语言理解中“当前词由其左侧上下文决定”的建模假设一致,并且可以将变长词序列稳定映射到定长向量。

\subsection{音频特征:TR 窗口切分与帧级池化}
音频特征提取遵循“波形分片—模型表征—窗口池化”的流程。整段音频以 16kHz 采样率读取后,按 TR 窗口切分为若干 chunk;每个 chunk 输入音频模型得到时间序列隐藏状态,再用 attention mask 对有效帧做平均池化得到单个向量。当前结果系统比较了 1TR、2TR、3TR、6TR 等多种窗口长度,目的在于检验更长的声学上下文是否有助于在 BOLD 延迟下提高可预测性。

\subsection{多模态特征:模型内部融合与 TR 对齐}
多模态模型的关键区别在于其输出不是“纯音频编码器的表示”,而是模型结构中显式对齐或融合了文本与音频信息后的表示。Whisper 属于编码器—解码器结构,本项目使用其编码器侧的表示作为与输入语音相关的表征来源,并对不同层进行比较;CLAP 同时包含音频与文本编码器,输出位于共享嵌入空间的音频表示,本项目将其视为多模态对齐框架下的表示来源,并在同样的 TR 窗口切分策略下进行评估。由于不同多模态模型对输入形式与采样率存在约束,当前报告仅讨论在 \texttt{results/} 中已经成功产出 corr map 的配置。

\chapter{编码模型、评价指标与输出结构}
\subsection{从 TR 特征到 BOLD:线性编码模型的形式化}
设某一类特征在 TR 级别上形成矩阵 $X\in\mathbb{R}^{T\times D}$,其中 $T$ 为有效 TR 数量、$D$ 为特征维度;对应被试的 fMRI ROI 信号为 $Y\in\mathbb{R}^{T\times V}$,其中 $V=360$ 为 ROI 数量。线性编码模型采用岭回归,对每个 ROI 同时求解权重矩阵 $W\in\mathbb{R}^{D\times V}$:
\begin{equation}
\hat{W}=\arg\min_{W}\ \lVert XW-Y\rVert_{2}^{2}+\alpha\lVert W\rVert_{2}^{2}.
\end{equation}
这里 $\alpha$ 为 $L_{2}$ 正则强度。岭回归的优势在于当 $D$ 较大且特征存在共线性时,仍能得到数值稳定的解,并在有限样本下缓解过拟合。当前工程中 \texttt{DEFAULT\_ALPHAS} 提供了若干候选正则强度,但由于 \texttt{DEFAULT\_KFOLD=1},实际训练并未进行 K 折交叉验证选择超参,而是在单次训练/测试划分下使用候选列表中的第一个 $\alpha$。因此,当前结果可被理解为一套“固定正则的线性基线”,其主要价值在于为不同特征与不同层提供统一且可追溯的对比基准。

这种“线性编码 + 严格时间对齐 + 在未见刺激上评估”的组合并非偶然,而是与自然叙事语义地图与集成建模工作在评价逻辑上保持一致。语义地图工作通过正则化线性回归在新故事上预测 fMRI,并据此把“可预测性”作为语义表征存在的证据 \cite{Huth2016}。大规模集成建模进一步强调统一评估协议的重要性,用以公平比较不同模型与不同任务,从而把差异尽可能归因到表示本身 \cite{Schrimpf2021}。与此同时,关于“预测编码是否是对齐来源”的争论提示我们,编码性能的提升既可能来自预测目标,也可能来自更一般的特征发现与结构归纳,因此需要在多模型、多层与多窗口条件下稳健比较并保持解释克制 \cite{Antonello2023,Kumar2024}。

\subsection{BOLD 延迟建模:PCA 降维与 FIR 延迟展开}
从预训练模型得到的 TR 特征往往维度较高,直接回归会带来计算成本与病态风险。因此,本项目在回归前对 TR 特征做 PCA 降维,默认保留 250 维(\texttt{DEFAULT\_PCA\_DIM=250})。随后,为显式建模 BOLD 延迟与时间扩散,本项目采用 FIR(finite impulse response)延迟展开:将每个 TR 的特征与若干个过去 TR 的特征按时间顺序拼接,形成扩展特征矩阵。当前默认设置为窗口长度 4、偏移 1(\texttt{DEFAULT\_FIR\_WINDOW=4}、\texttt{DEFAULT\_FIR\_OFFSET=1}),这意味着在预测某一 TR 的 fMRI 时,模型可以使用从较早 TR 开始、覆盖若干步历史的刺激表示,从而在不引入非线性结构的前提下捕捉响应延迟。

\subsection{数据划分与多被试汇总}
编码模型以每个被试为单位独立训练与评估:对每个被试的 $(X,Y)$ 在时间轴上做截断以去除边界 TR,然后按时间顺序切分为训练段与测试段(当前实现默认测试比例为 0.2)。在测试段上计算预测信号与真实信号的相关系数,得到长度为 360 的相关向量(corr map)。为得到多被试的总体性能,项目对每个被试的 corr map 求均值作为该被试的总体分数,再对所有被试分数计算均值与标准差并写入 \texttt{log.txt}。因此,报告中“平均值 $\pm$ 标准差”对应的是跨被试的统计,而不是跨折的统计。

\subsection{评价指标与 corr map 的含义}
评价指标为 Pearson 相关系数。对第 $v$ 个 ROI,设测试段真实信号为 $y_{v}$、预测信号为 $\hat{y}_{v}$,则相关为
\begin{equation}
r_{v}=\frac{\sum_{t}(\hat{y}_{v,t}-\overline{\hat{y}_{v}})(y_{v,t}-\overline{y_{v}})}{\sqrt{\sum_{t}(\hat{y}_{v,t}-\overline{\hat{y}_{v}})^{2}}\sqrt{\sum_{t}(y_{v,t}-\overline{y_{v}})^{2}}}.
\end{equation}
将所有 ROI 的 $r_{v}$ 组成向量即可得到 corr map。项目保存的 \texttt{corr\_layer*.npy} 即为该向量,长度为 360,前 180 对应左半球 ROI,后 180 对应右半球 ROI。可视化时,\texttt{src/viz.py} 读取 HCP-MMP 的 ROI 标签文件,将 ROI 相关值映射回 fsaverage 表面顶点并绘制,输出以左右半球的外侧与内侧视图组成的四联图,保证角度稳定且信息密集。

\subsection{输出文件结构与可追溯性}
本项目输出结构以 \texttt{results/} 为根。文本、音频、多模态的线性编码结果分别位于 \texttt{results/text/}、\texttt{results/audio/}、\texttt{results/multimodal/};每个配置目录包含 \texttt{log.txt} 与若干 \texttt{corr\_layer*.npy}。融合结果位于 \texttt{results/fusion/},其中每个融合对目录包含融合日志与多个融合 corr map 文件(例如 \texttt{corr\_t9\_a6\_ctx200\_tr1.npy})。报告中的统计图由 \texttt{report/scripts/make\_figures.py} 从 \texttt{results/summary.csv} 与 \texttt{results/roi.csv} 生成,脑图由 \texttt{src/run\_plot\_corr\_maps.py} 从 corr map 生成并保存到 \texttt{report/figures/brainmaps/}。该设计使得报告中每一张图都能追溯回唯一的源文件路径,便于复核与增量补充实验。

\begin{figure}[H]
\centering
\begin{tikzpicture}[node distance=0.9cm, font=\small, scale=0.95, transform shape]
\node[draw, rounded corners, fill=BrandColor!4, inner sep=6pt] (stim) {刺激(文本/音频)};
\node[draw, rounded corners, fill=BrandColor!4, right=1.0cm of stim, inner sep=6pt] (feat) {预训练模型特征(多层)};
\node[draw, rounded corners, fill=BrandColor!4, right=1.0cm of feat, inner sep=6pt] (tr) {TR 聚合与对齐};
\node[draw, rounded corners, fill=BrandColor!4, right=1.0cm of tr, inner sep=6pt] (pca) {PCA(250)};

\node[draw, rounded corners, fill=BrandColor!4, below=1.0cm of feat, inner sep=6pt] (fir) {FIR(4,1)};
\node[draw, rounded corners, fill=BrandColor!4, right=1.0cm of fir, inner sep=6pt] (ridge) {岭回归};
\node[draw, rounded corners, fill=BrandColor!4, right=1.0cm of ridge, inner sep=6pt] (corr) {corr map(360)};

\draw[->, thick] (stim) -- (feat);
\draw[->, thick] (feat) -- (tr);
\draw[->, thick] (tr) -- (pca);
\draw[->, thick] (pca) |- (fir);
\draw[->, thick] (fir) -- (ridge);
\draw[->, thick] (ridge) -- (corr);
\end{tikzpicture}
\caption{线性编码建模流水线概览。为避免版面溢出,示意图采用两行布局,但顺序与实现一致。}
\end{figure}

\section{实验结果(一):文本模型}
文本模型结果来自 \texttt{results/summary.csv} 中 \texttt{/text/} 条目及对应的 \texttt{corr\_layer*.npy}。本次文本特征采用固定 200 token 上下文窗口,并将词级表示在 TR 内取平均得到 TR 级特征。该设置在方法上提供了稳定对照,但也意味着文本表征受限于窗口长度与 TR 内词数波动。由于本文不引入超出已完成实验的推断,结果部分仅对当前设置下的数值与空间分布作出可追溯陈述。

三种文本模型的最佳层均值相关分别为:RoBERTa-base 在 layer4 达到 $0.0153 \pm 0.0160$,GPT-2 在 layer12 达到 $0.0107 \pm 0.0137$,BERT-base 在 layer6 达到 $0.0084 \pm 0.0161$。图 \ref{fig:text_best} 以柱状图形式对比了三种模型的最佳层表现。就数量级而言,文本模型在本任务中的可预测性显著低于音频与多模态模型(后文将给出 0.08--0.09 量级的强基线),因此文本结果在本文中主要承担“纯语义/文本基线”的作用,用于在相同评估协议下刻画语义特征在当前数据与时间尺度下的可预测上限。

空间分布方面,图 \ref{fig:text_montage} 将三种文本模型的最佳层相关图并置,便于观察在同一绘图视角与同一色标下的差异。为避免把跨模型差异误读为绘图设置差异,本图的每一张脑图都来自相同的 corr map 到皮层表面映射流程。进一步地,图 \ref{fig:text_roberta_map}、图 \ref{fig:text_gpt2_map} 与图 \ref{fig:text_bert_map} 分别展示三种模型在其最佳层的单独脑图,从而便于在后续讨论 ROI 偏好与空间模式时引用。尽管文本模型整体相关较低,图 \ref{fig:text_roberta_roi} 的 ROI Top20 仍能为“哪些区域在当前框架下更容易被文本语义预测”提供定量入口,并可与音频/多模态的 ROI 分布形成对照。

\InsertFig{text_best.png}{0.86\linewidth}{文本模型最佳层的全脑均值相关系数(从 \texttt{results/summary.csv} 聚合)。}{fig:text_best}
\InsertFig{text_class_montage.png}{0.96\linewidth}{文本模型最佳层脑图对照:RoBERTa(win200,layer4)、GPT-2(win200,layer12)、BERT(win200,layer6)。}{fig:text_montage}
\InsertFig{text_roberta-base_win200_corr_layer4.png}{0.92\linewidth}{RoBERTa-base(win200,layer4)相关图可视化。}{fig:text_roberta_map}
\InsertFig{text_gpt2_win200_corr_layer12.png}{0.92\linewidth}{GPT-2(win200,layer12)相关图可视化。}{fig:text_gpt2_map}
\InsertFig{text_bert-base-uncased_win200_corr_layer6.png}{0.92\linewidth}{BERT-base(win200,layer6)相关图可视化。}{fig:text_bert_map}
\InsertFig{roi_roberta_l4.png}{0.86\linewidth}{RoBERTa-base(win200,layer4)对应相关图的 ROI Top20(从 \texttt{results/roi.csv} 聚合)。}{fig:text_roberta_roi}

\section{实验结果(二):音频模型}
音频模型结果覆盖三种模型(wav2vec2、WavLM、HuBERT)以及四种 TR 窗口长度(1TR、2TR、3TR、6TR)。在音频侧,TR 窗口长度不仅改变了输入模型的时间上下文,也决定了 TR 级特征所汇聚的声学信息范围,因此它在自然故事听觉范式下具有直接的可解释意义。本文在统一的评估协议下比较不同窗口与不同层的表现,从而把“时间尺度”作为与“层级位置”并列的系统变量。

从模型整体最佳表现看,WavLM-base-plus 的最优配置出现在 6TR、layer9,均值相关为 $0.0916 \pm 0.0405$;wav2vec2-base-960h 的最优配置出现在 6TR、layer9,均值相关为 $0.0895 \pm 0.0320$;HuBERT-base-ls960 的最优配置出现在 6TR、layer9,均值相关为 $0.0823 \pm 0.0346$。图 \ref{fig:audio_best} 汇总了三种音频模型的最佳层表现,可以看到音频表征在当前数据与评估框架下构成全局强基线,其数量级显著高于文本模型。

TR 窗口效应在三种音频模型上表现为一致的单调提升。以“每个窗口内的最佳层”为代表,WavLM 在 1TR、2TR、3TR、6TR 的最佳均值相关分别为 $0.0316$(layer9)、$0.0450$(layer4)、$0.0583$(layer9)、$0.0916$(layer9);wav2vec2 分别为 $0.0274$(layer1)、$0.0382$(layer1)、$0.0569$(layer9)、$0.0895$(layer9);HuBERT 分别为 $0.0326$(layer4)、$0.0452$(layer4)、$0.0556$(layer4)、$0.0823$(layer9)。图 \ref{fig:audio_tr_trend} 将这一趋势可视化为曲线,表明更长的声学上下文在当前编码框架下显著提高可预测性。这一现象与自监督语音表征“在更长上下文内形成更稳定结构表示”的观点相一致 \cite{Baevski2020,Millet2023},但本文在结果部分不将其解释为机制因果,仅将其作为在统一评估协议下的稳健经验结论。

空间分布方面,图 \ref{fig:audio_montage} 并置了三种音频模型在其最优配置下的脑图,从而在同一视角与色标下比较其皮层分布。图 \ref{fig:audio_wavlm_map} 展示全局最优音频配置(WavLM 6TR layer9)的单独脑图,图 \ref{fig:audio_w2v2_map} 与图 \ref{fig:audio_hubert_map} 分别展示 wav2vec2 与 HuBERT 的最优配置脑图。ROI 层面,图 \ref{fig:audio_wavlm_roi} 给出 WavLM 最优配置的 ROI Top20,为后续讨论“不同脑区对不同模态表征的偏好”提供对照基线。

\InsertFig{audio_best.png}{0.86\linewidth}{音频模型最佳层的全脑均值相关系数(从 \texttt{results/summary.csv} 聚合)。}{fig:audio_best}
\InsertFig{audio_tr_window_trend.png}{0.92\linewidth}{音频模型在不同 TR 窗口下的最佳层性能趋势(从 \texttt{results/summary.csv} 聚合)。}{fig:audio_tr_trend}
\InsertFig{audio_class_montage.png}{0.96\linewidth}{音频模型最优配置脑图对照:WavLM、wav2vec2、HuBERT。}{fig:audio_montage}
\InsertFig{audio_microsoft_wavlm-base-plus_6TR_corr_layer9.png}{0.92\linewidth}{WavLM-base-plus(6TR,layer9)相关图可视化。}{fig:audio_wavlm_map}
\InsertFig{audio_facebook_wav2vec2-base-960h_6TR_corr_layer9.png}{0.92\linewidth}{wav2vec2-base-960h(6TR,layer9)相关图可视化。}{fig:audio_w2v2_map}
\InsertFig{audio_facebook_hubert-base-ls960_6TR_corr_layer9.png}{0.92\linewidth}{HuBERT-base-ls960(6TR,layer9)相关图可视化。}{fig:audio_hubert_map}
\InsertFig{roi_wavlm6tr_l9.png}{0.86\linewidth}{WavLM-base-plus(6TR,layer9)对应相关图的 ROI Top20(从 \texttt{results/roi.csv} 聚合)。}{fig:audio_wavlm_roi}

\section{实验结果(三):多模态模型}
多模态模型结果来自 \texttt{results/summary.csv} 中 \texttt{/multimodal/} 条目及其对应的 corr map。与纯音频模型相比,多模态模型在训练目标或表示空间上显式引入了跨模态约束,因此它们在自然故事听觉任务中的表现能够回答一个具体问题:在统一的编码评估框架下,多模态表征是否在保持强音频基线的同时,进一步靠近语义系统的可预测模式。本文在当前已生成结果范围内对 Whisper 与 CLAP 的可用输出进行比较,并把其结果与音频基线对齐在同一量纲下呈现。

从整体最佳表现看,Whisper-base 的最优配置出现在 6TR、layer2,均值相关为 $0.0889 \pm 0.0321$;Whisper-small 的最优配置出现在 6TR、layer9,均值相关为 $0.0844 \pm 0.0324$;CLAP 的最优配置在当前结果中对应到 layer1,并在 6TR 时达到 $0.0788 \pm 0.0344$。图 \ref{fig:mm_best} 汇总了多模态模型的最佳层表现,并可与图 \ref{fig:audio_best} 的音频模型最佳层直接对照。就数值而言,Whisper-base 的最优均值与 wav2vec2/WavLM 的 6TR 最优结果处在同一量级,说明在自然故事听觉范式下,Whisper 的内部表征能够提供与强音频基线相近的可预测性,但并未显著超过最强音频模型。

TR 窗口效应在多模态模型上同样呈现一致的单调提升。以“每个窗口内的最佳层”为代表,Whisper-base 在 1TR、2TR、3TR、6TR 的最佳均值相关分别为 $0.0270$(layer1)、$0.0408$(layer4)、$0.0579$(layer4)、$0.0889$(layer2);Whisper-small 分别为 $0.0289$(layer9)、$0.0417$(layer9)、$0.0543$(layer12)、$0.0844$(layer9);CLAP 分别为 $0.0231$、$0.0351$、$0.0509$、$0.0788$(均对应 layer1)。图 \ref{fig:mm_tr_trend} 将这一趋势可视化,表明多模态模型同样依赖更长的声学上下文来提高对齐强度。结合自监督语音表征的层级对齐观点 \cite{Baevski2020,Millet2023},这一现象提示“时间尺度”在自然故事范式下对音频相关表征具有普遍影响,但其是否对应更高层语义整合仍需要在更丰富的语义控制实验中才能区分。

空间分布方面,图 \ref{fig:mm_montage} 并置了多模态模型的代表性脑图,用于与音频与文本结果在同一视角下比较。图 \ref{fig:mm_whisper_base_map} 展示 Whisper-base 最优配置(6TR,layer2)的单独脑图,图 \ref{fig:mm_whisper_small_map} 展示 Whisper-small 的最优配置(6TR,layer9)。ROI 层面,图 \ref{fig:mm_whisper_roi} 给出 Whisper-base 最优配置的 ROI Top20,用于在讨论中分析其与音频基线在区域偏好上的相同与差异。需要说明的是,CLAP 的脑图在当前本地绘图目录中未形成可引用的图像文件,因此本文仅在数值层面对 CLAP 的可预测性进行报告,并将其空间分析留作后续补齐绘图输出后的扩展。

\InsertFig{multimodal_best.png}{0.86\linewidth}{多模态模型最佳层的全脑均值相关系数(从 \texttt{results/summary.csv} 聚合)。}{fig:mm_best}
\InsertFig{multimodal_tr_window_trend.png}{0.92\linewidth}{多模态模型在不同 TR 窗口下的最佳层性能趋势(从 \texttt{results/summary.csv} 聚合)。}{fig:mm_tr_trend}
\InsertFig{multimodal_class_montage.png}{0.96\linewidth}{多模态模型代表性脑图对照(由本地已生成的脑图组合)。}{fig:mm_montage}
\InsertFig{multimodal_openai_whisper-base_6TR_corr_layer2.png}{0.92\linewidth}{Whisper-base(6TR,layer2)相关图可视化。}{fig:mm_whisper_base_map}
\InsertFig{multimodal_openai_whisper-small_6TR_corr_layer9.png}{0.92\linewidth}{Whisper-small(6TR,layer9)相关图可视化。}{fig:mm_whisper_small_map}
\InsertFig{roi_whisper6tr_l2.png}{0.86\linewidth}{Whisper-base(6TR,layer2)对应相关图的 ROI Top20(从 \texttt{results/roi.csv} 聚合)。}{fig:mm_whisper_roi}

\chapter{文本+音频融合结果:实现细节、当前覆盖范围与可视化}
融合实验的输出位于 \texttt{results/fusion/}。与单模态结果使用 \texttt{corr\_layer*.npy} 命名不同,融合结果的文件名包含文本层、音频层、文本上下文窗口与音频 TR 窗口,例如 \texttt{corr\_t9\_a9\_ctx200\_tr1.npy}。融合的核心思想是把文本与音频在 TR 级对齐后进行特征拼接,然后在同一套 PCA+FIR+岭回归框架下评估其对 fMRI 的可预测性。实现上,\texttt{src/run\_multimodal\_fusion.py} 对文本与音频分别做标准化(\texttt{StandardScaler}),将两者在维度上拼接得到融合特征,再在拼接后的空间中执行 PCA(默认 250 维),最后做 FIR 延迟展开并回归到 360 个 ROI。该过程与单模态的区别在于:单模态的 PCA 发生在原始特征空间,而融合的 PCA 发生在拼接后的联合空间,因此其主成分同时受到文本与音频方差结构的影响。

当前仓库中可追溯的融合记录显示,融合并未覆盖所有文本模型、所有音频模型、所有 TR 窗口与所有层的笛卡尔积,而是受到“是否已存在对应的特征文件”的约束。具体而言,\texttt{results/fusion/} 下已存在三个融合对目录(例如 \texttt{gpt2\_\_microsoft\_wavlm-base-plus}),其中每个目录包含 \texttt{log.txt} 与若干融合 corr map。由于音频最强结果集中在 6TR,而融合记录主要出现在更短的 TR 窗口条件下,因此在数值对照时必须避免将“不同窗口条件下的融合结果”与“6TR 条件下的音频最优结果”作强结论比较。本报告在此严格陈述:在当前已生成的融合日志中,最好的条目来自 \texttt{gpt2} 与 \texttt{microsoft/wavlm-base-plus} 的组合,其均值相关约为 0.0302(标准差约 0.0171),对应配置为 \texttt{text\_layer=9}、\texttt{audio\_layer=9}、\texttt{ctx\_words=200}、\texttt{tr\_win=1}。这一水平明显低于音频与多模态在 6TR 条件下取得的 0.08--0.09 量级结果,但该差距是否来自融合机制本身、来自窗口不匹配、或来自组合覆盖不充分,均需要在后续扩展相同窗口条件下的融合实验后才能判断。

为了把融合的空间分布纳入“图文一致”的可追溯链路,图 \ref{fig:fusion_map_best} 展示上述最优融合配置的脑图可视化。该图像的作用是让后续工作可以在相同视角下对比融合、音频与多模态的空间分布差异,并检查融合是否在部分 ROI 上更接近文本或音频的模式。

此外,为了给出融合目录中一个明确可复现的示例,图 \ref{fig:fusion_map_example} 额外展示一个不同层组合(\texttt{t9\_a6\_ctx200\_tr1})的脑图,便于观察在相同模型对与相同窗口下,仅改变层组合时空间分布可能发生的变化。

\begin{table}[H]
\centering
\caption{融合(gpt2 + WavLM,ctx=200,tr=1)在已保存日志中的部分代表性条目。数值直接来自 \texttt{results/fusion/gpt2\_\_microsoft\_wavlm-base-plus/log.txt}。}
\begin{tabular}{cccc}
\toprule
文本层 & 音频层 & 均值相关 & 标准差 \\
\midrule
9 & 9 & 0.0302 & 0.0171 \\
12 & 9 & 0.0301 & 0.0157 \\
9 & 12 & 0.0292 & 0.0168 \\
9 & 1 & 0.0283 & 0.0154 \\
6 & 4 & 0.0274 & 0.0189 \\
\bottomrule
\end{tabular}
\end{table}

\InsertFig{fusion_gpt2__microsoft_wavlm-base-plus_corr_t9_a9_ctx200_tr1.png}{0.92\linewidth}{最优融合配置:gpt2(layer9)+ WavLM(layer9),ctx=200,tr=1 的相关图可视化(由 src/run\_plot\_corr\_maps.py 从 results/fusion 的 corr\_*.npy 绘制)。}{fig:fusion_map_best}
\InsertFig{fusion_gpt2__microsoft_wavlm-base-plus_corr_t9_a6_ctx200_tr1.png}{0.92\linewidth}{融合示例:gpt2(layer9)+ WavLM(layer6),ctx=200,tr=1 的相关图可视化(由 src/run\_plot\_corr\_maps.py 从 results/fusion 的 corr\_*.npy 绘制)。}{fig:fusion_map_example}

\section{时间尺度效应(TR 窗口)}
在音频与多模态结果中,TR 窗口增大会带来稳定的性能上升(图 \ref{fig:audio_tr_trend}、图 \ref{fig:mm_tr_trend}),融合结果也呈现同向趋势(图 \ref{fig:fusion_tr_trend})。该现象说明:在自然故事听觉范式下,TR 级输入的时间聚合尺度是影响编码性能的主导因素之一。以音频模型为例,WavLM 的最佳均值相关从 1TR 的 $0.0316$(layer9)上升到 6TR 的 $0.0916$(layer9);wav2vec2 从 1TR 的 $0.0274$(layer1)上升到 6TR 的 $0.0895$(layer9);HuBERT 从 1TR 的 $0.0326$(layer4)上升到 6TR 的 $0.0823$(layer9)。多模态 Whisper-base 的最佳均值相关从 1TR 的 $0.0270$(layer1)上升到 6TR 的 $0.0889$(layer2),CLAP 同样随窗口上升。窗口效应在不同模型家族中同时出现,表明性能差异的比较必须以相同 TR 窗口为前提,否则“窗口长度”本身会掩盖模型与层级差异。

\section{层级趋势(跨层对齐)}
为了更加准确的对比模型的层级差异,我们选择的模型全部是 12 层,其目的不是只报告“最优层”,而是用层级曲线回答“表征在模型内部的哪些抽象层级更容易与大脑对齐”。我们把同一评估协议下的层级结果按层汇总并可视化:文本侧为 win200(图 \ref{fig:text_layer_trend}),音频侧为 6TR(图 \ref{fig:audio_layer_trend})。

文本侧的层级曲线呈现明显的“模型依赖峰值层”:RoBERTa 的最优出现在中层(layer4,$0.0153 \pm 0.0160$),BERT 的最优出现在中高层(layer6,$0.0084 \pm 0.0161$),GPT-2 的最优出现在更高层(layer12,$0.0107 \pm 0.0137$)。同一“文本—fMRI”对齐任务上,不同语言模型把有效信息分配到不同层级位置,这使得跨模型比较必须同时考虑“模型差异”和“层级位置差异”,否则会把“峰值层位置变化”误当作“模型好坏变化”。 考虑到我们限制了模型的层数,所以关于文本模型层级趋势的结论仍然有限,如果纳入更多参数量级的模型(如 GPT-3、LLaMA 等),可能会观察到不同的层级趋势。

音频侧的层级曲线更集中:在 6TR 条件下,三种音频模型的峰值层都落在中高层(wav2vec2/WavLM/HuBERT 的最佳层分别为 layer9/layer9/layer9,对应 $0.0895$/$0.0916$/$0.0823$)。在较短窗口下,wav2vec2 的最佳层会更偏低层(例如 1TR/2TR 最佳层为 layer1),而窗口增大后峰值层移动到中高层(3TR/6TR 最佳层为 layer9)。层级曲线与窗口效应结合起来给出一个直接的结构性结论:在更长时间聚合尺度下,与 BOLD 更稳定对齐的表征更集中在“跨时间整合更强”的中高层输出,而不是仅反映局部输入变化的低层表示。

\InsertFig{text_layer_trend_win200.png}{0.92\linewidth}{文本模型:不同层的对齐性能(win200,均值$\pm$标准差,从 \texttt{results/summary.csv} 聚合)。}{fig:text_layer_trend}
\InsertFig{audio_layer_trend_6tr.png}{0.92\linewidth}{音频模型:不同层的对齐性能(6TR,均值$\pm$标准差,从 \texttt{results/summary.csv} 聚合)。}{fig:audio_layer_trend}

\section{模态差异(文本 vs 音频/多模态)}
在同一评估协议下,文本模型的整体相关量级显著低于音频与多模态:文本最佳约为 $0.015$(图 \ref{fig:text_best}),而音频与多模态在 6TR 条件下达到 $0.08\text{--}0.09$(图 \ref{fig:audio_tr_trend}、图 \ref{fig:mm_tr_trend})。这一差异与“特征对齐链路的压缩程度”一致:音频/多模态表征直接来自连续语音片段并在 TR 窗口上聚合,保留了与刺激同构的时间结构;文本表征来自转写序列,对齐到 TR 时必须把 token/词级序列在 TR 内再次聚合(本项目采用 TR 内平均池化),从而把 TR 内词数波动与边界效应折叠进单一均值向量。对齐链路上的信息压缩越强,进入回归的可用信号越弱,最终在相关指标上呈现量级差异。

另一条与模态差异一致的事实是:语音信号本身承载语言单位相关的结构(音素、音节、韵律、停顿与重音等),这些结构与词边界、句法分组和语义重点相关。因此强音频/语音表征不仅反映声学变化,也包含对语言结构的刻画;在听觉范式下,这种“声学+语言结构”的混合表征能够在 TR 聚合后形成更强、更稳定的预测信号,这为音频/多模态模型在总体量级上领先文本基线提供了直接解释链路。

\section{非线性对照(文本 aligned 特征)}
非线性读出器的对照结果与上述“特征主导”结论一致:核岭回归(RBF)的最优为 $0.0141 \pm 0.0186$(RoBERTa aligned\_layer1),低于线性文本基线的最优 $0.0153 \pm 0.0160$(RoBERTa layer4)。该对照在数值上体现为:即使把读出器从线性换成非线性形式,整体量级仍停留在 $10^{-2}$,并未跨越文本侧的瓶颈。

该现象与文本特征进入回归前的两次压缩过程一致:第一,token/词级序列表征在对齐到 fMRI 时被 TR 内平均池化为单个向量;第二,文本上下文窗口固定为 200 token,使跨 TR 的长程语义依赖在输入阶段被截断。压缩后的 TR 级文本特征在“时间结构”和“语义差异”两个维度上都更稀疏;当输入侧有效信息量不足时,读出器形式(线性/非线性)不会成为决定性因素,性能更取决于特征提取与对齐的质量与丰富度。


\section{结论、局限与后续可扩展方向}
在当前已经完成并保存为结果文件的实验范围内,编码模型的可预测性呈现出清晰且可复现的模态差异。文本模型在 \texttt{win200} 的设置下整体相关较低,即便在各自最佳层,其跨被试均值相关仍处于 $10^{-2}$ 量级;音频模型在较长 TR 窗口下达到约 0.08--0.09 的均值相关,并且不同音频模型在窗口增大时都呈现一致的性能提升趋势;多模态模型中 Whisper-base 在 6TR 条件下取得与强音频基线接近的性能,CLAP 在当前完成范围内略低但同样随窗口增长而提升。上述结论既体现在 \texttt{results/summary.csv} 的均值对比上,也在脑图与 ROI Top20 图上得到一致支持:强模型不仅抬升全脑均值,也使多个 ROI 的相关分布整体上移。

融合实验已经形成大规模可追溯输出。与此前仅能展示少量融合示例不同,当前融合在 \texttt{ctx=200} 下覆盖 \texttt{tr\_win=1/2/3} 三种窗口、3 个文本模型与 3 个音频模型以及多层组合,融合日志可解析记录达到 540 条。融合的全局最优出现在 \texttt{tr\_win=3},对应 \texttt{bert-base-uncased + microsoft/wavlm-base-plus} 的层组合(\texttt{t6\_a9\_ctx200\_tr3}),跨被试均值相关达到 0.0535。该数值显著高于文本单模态并在融合内部呈现清晰的窗口效应与层交互结构,但仍低于音频与多模态在 6TR 条件下的 0.08--0.09 量级结果。因此,对“融合是否优于强音频基线”的最终回答仍取决于是否能在更长窗口(例如 6TR)与更一致的特征覆盖条件下完成对等比较。

本报告的第二个重要结论不是某个单一数值,而是本项目的可追溯链路已经被建立并可稳定复用。每一条结果都能从报告中的统计图或脑图回溯到唯一的源文件路径,进一步回溯到生成该文件的脚本与配置,从而使后续扩展实验(例如增加更多文本模型、补齐多模态模型、系统搜索窗口与池化策略、或加入显著性检验)可以在不破坏现有结构的前提下增量进行。与此同时,本报告也明确存在当前结果目录所决定的局限:其一,回归超参与划分策略在当前配置下未进行交叉验证选择,因此结果更适合用于特征对比的基线,而不适合用于对绝对性能做过度外推;其二,ROI 编号尚未映射到解剖名称,限制了对“语义偏好性”的命名解释;其三,非线性编码模型在当前结果目录中未形成与线性结果同结构的 corr map 输出,因此无法纳入同一套图像证据链进行比较。

在上述边界内,本报告已经完成了对现有结果的系统整理:对多模型、多层、多窗口的线性编码结果给出数值对比;对文本、音频、多模态三个类别分别提供“类别脑图对照”与“最优配置脑图”;在 ROI 层面展示代表性模型的 Top20 分布并讨论其可解释性与局限。后续工作的关键并不是对文字叙述做任何“补写”,而是在现有流水线中补齐缺失的实验维度,使这些章节能够在同一模板下自然扩展并与综述合并。


% ===================== 附录:完整原始数据(不通过脚本生成) =====================
\appendix
\chapter{附录:原始结果文件(逐行原样嵌入)}
为满足“所有已完成实验数据均在报告中可追溯”的要求,本附录直接嵌入关键 CSV 与融合日志的原始内容。此处不对原始文件进行二次处理,也不通过脚本生成表格,确保内容与磁盘一致。

\section{results/summary.csv}
\VerbatimInput[fontsize=\scriptsize]{results/summary.csv}

\section{results/fusion/gpt2\_\_microsoft\_wavlm-base-plus/log.txt}
\VerbatimInput[fontsize=\scriptsize]{results/fusion/gpt2__microsoft_wavlm-base-plus/log.txt}

\section{results/fusion/gpt2\_\_facebook\_wav2vec2-base-960h/log.txt}
\VerbatimInput[fontsize=\scriptsize]{results/fusion/gpt2__facebook_wav2vec2-base-960h/log.txt}

\section{results/roi.csv}
由于 \texttt{results/roi.csv} 为按 MMP ROI 展开的全量记录,行数较大,直接嵌入会导致 \TeX{} 内存耗尽。本报告已在正文中通过 ROI Top20 图对该文件进行了统计使用;原始文件保持在仓库的 \texttt{results/roi.csv} 路径下,供复核与后续命名映射使用。

\end{document}
